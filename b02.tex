\begin{poem}
\sequencetitle{DÚVIDAS APÓCRIFAS DE MARIANNE MOORE}
\begin{stanza}
Sempre evitei falar de mim,\verseline
falar-me. Quis falar de coisas.\verseline
Mas na seleção dessas coisas\verseline
não havera um falar de mim?
\end{stanza}
\begin{stanza}
Não havera nesse pudor\verseline
de falar-me uma confissão,\verseline
uma indireta confissao,\verseline
pelo avesso, e sempre impudor?
\end{stanza}
\begin{stanza}
A casa de que se falar\verseline
ate onde está pura ou impura?\verseline
Ou sempre se impõe, mesmo impura-\verseline
mente, a quem dela quer falar?
\end{stanza}
\begin{stanza}
Como saber, se há tanta coisa\verseline
de que falar ou não falar?\verseline
E se o evitá-la, o não falar,\verseline
é forma de falar da coisa?
\end{stanza}
\attribution{JCMN}
\end{poem}