\begin{poem}
\begin{stanza}
Eu não quero outra coisa\verseline
só quero uma
\end{stanza}
\begin{stanza}
Que isso que queima fosse verdade!\verseline
Se fosse. Não sei\verseline
o que seria
\end{stanza}
\poemsubtitle{b.}
\begin{stanza}
só uma coisa \qquad a maior delas\verseline
a coisaagora \qquad verdadiamante\verseline
ou melhor \qquad se chama\verseline
chama \qquad coisa que queima\verseline
há duas \qquad uma\verseline
forte e antiga \qquad outra\verseline
branda \qquad nem se desenvolveu\verseline
\qquad toda boa lâmina\verseline
é lisa \qquad e doce\verseline
isso não \qquad é so e apenas\verseline
acidente\verseline
\qquad \qquad tropeço
\end{stanza}
\poemsubtitle{c.}
\begin{stanza}
a coisa que queima se chama\verseline
\qquad \qquad \qquad \qquad \qquad chama\verseline
coisa que está no centro\verseline
\qquad \qquad \qquad \qquad \qquad dentro das entranhas\verseline
inflamada entramada nas fibras\verseline
entranhada e estranha\verseline
\qquad há essa coisa sem palavra\verseline
\qquad e sem chamar \qquad so chamada\verseline
\qquad por um algo não nome\verseline
\qquad desejo mortalmente\verseline
o nome da agora coisa é também absurdo\verseline
como pode o fogo?\verseline
\qquad \qquad \qquad \qquad \qquad o que queima até consumir\verseline
ora se isso destrói \qquad como desejar com\verseline
chama? \qquad como querer até a morte?\verseline
se depois não se tem?
\end{stanza}
\poemsubtitle{d.}
\begin{stanza}
a agora coisa que levo se chama\verseline
chama\verseline
coisa entramada na entranha\verseline
entranhada nas fibras\verseline
tem uma estranha palavra não nome\verseline
desejar até a morte
\end{stanza}
\poemsubtitle{e.}
\begin{stanza}
a coisa de que falo é a coisa\verseline
que faz eco no escuro\verseline
a coisa de que falo é\verseline
o instrumento da vitoria\verseline
\qquad nome de mulher\verseline
isto conseguido sem esforço\verseline
isto de que falo\verseline
a fala
\end{stanza}
\poemsubtitle{f.}
\begin{stanza}
a coisa de que falo\verseline
é um vômito contido\verseline
essa coisa que queima se chama\verseline
chama da entranha\verseline
o estranho não nome\verseline
do desejar calado
\end{stanza}
\rotatebox[origin=c]{180}
{
\begin{minipage}[t]{\textwidth}
\begin{stanza}
essa coisa de que falo\verseline
é também o que desejo\verseline
e calo\verseline
uso uma palavra\verseline
não nome\verseline
para referir o que está dentro\verseline
e queima\verseline
e é só uma pena\verseline
um peso
\end{stanza}
\end{minipage}
}\\
\poemsubtitle{g.}
\begin{stanza}
aquilo de que falo\verseline
é o que tanto sabem\verseline
não preciso dizer um nome\verseline
para mostrar o que melhor\verseline
se compreende no escuro\verseline
falo do que esta do outro lado\verseline
e é inalcançável
\end{stanza}
\rotatebox[origin=c]{180}
{
\begin{minipage}[t]{\textwidth}
\begin{stanza}
ou é alcançável por alguns não mim\verseline
\qquad ADENDUM\verseline
\qquad APPENDIX\verseline 
a palavra escrita no espelho\verseline
as palavras que não fazem\verseline
sentido juntas \qquad o que não trabalho\verseline
foi outro que disse \qquad eu queria ter es-\verseline
crito um livro \qquad DO DESEJO\verseline
do deserto do não beijo essa palavra
\end{stanza}
\end{minipage}
}\\
\poemsubtitle{h.}
\begin{stanza}
essa palavra que eu não quero dizer\verseline
\qquad DESEJO\verseline
essa palavra que queima em mim e se chama\verseline
chama \qquad ainda que estivesse no deserto\verseline
mais seco \qquad minha boca estaria úmida\verseline
de querer \qquad ainda que estivesse exilado\verseline
na sibéria \qquad meu corpo estaria quente\verseline
de querer com fogo \quad fervor de sangue ardente\verseline
ou melhor querer com fervor de crente\verseline
e uma febre de doente
\end{stanza}
\poemsubtitle{h.0.}
\begin{stanza}
essa palavra que não quero dizer\verseline
\qquad DESEJO\verseline
isso que queima e se chama\verseline
chama \qquad reescrevi todos os livros\verseline
que tinha lido\verseline
para que alguém me saiba\verseline
ponho meu nome no que não é meu\verseline
nem eu mesmo\verseline
me reconheço \qquad só sei\verseline
de uma coisa entranhada\verseline
nas fibras de mim\verseline
usei palavras graves\verseline
gravadas a ferro\verseline
mas so quero saber de entrar\verseline
cada vez mais no labirinto\verseline
eu quero eu quero eu quero\verseline
nem eu mesmo\verseline
me enxergo no que sou\verseline
olho nas caras dos outros\verseline
tentando me encontrar\verseline
mas tudo isso no princípio apenas\verseline
queria dizer \qquad olhos azuis\verseline
depois \qquad tênis vermelhos\verseline
depois \qquad nada \quad que é o que\verseline estava
e ainda está\verseline
e não se escapa\verseline
nunca.
\end{stanza}
\poemsubtitle{h.1.}
\begin{stanza}
mas essa palavra que não quero dizer\verseline
\qquad DESEJO\verseline
isso que queima e se chama\verseline
chama\verseline
são tênis vermelhos em sua dança\verseline
e calças tambem\verseline
e não é ninguém\verseline
\qquad é a dança vermelha\verseline
\qquad dos tênis nos pés\verseline
\qquad \qquad calcanhares sao nomes\verseline
\qquad \qquad que giram o corpo\verseline
\qquad \qquad \qquad eu usaria até o termo\verseline
\qquad \qquad \qquad gracioso\verseline
\qquad \qquad \qquad \qquad ou outro um\verseline
\qquad \qquad \qquad \qquad que não encontro\verseline
isso quero e quero e o que eu quero\verseline
se chama\verseline
fogo inflamado \qquad \qquad dentro de mim\verseline
por essa dança
\end{stanza}
\poemsubtitle{h.2.}
\begin{stanza}
ou por outra dança\verseline
que são os olhos \qquad o uso próprio\verseline
de dois pra la dois pra cá \quad a expressão\verseline
perfeitamente burra\verseline
o obstáculo à leitura do que seja\verseline
em olhos \qquad é o desviar\verseline
sem nunca encontrar de novo
\end{stanza}
\poemsubtitle{h.3.}
\begin{stanza}
mas isso que quero\verseline
e temo\verseline
esse bale da cor\verseline
de notas baixas\verseline
isso não me infala\verseline
e não \quad não se chama\verseline
chama \quad pelo contrario\verseline
se apela por outra palavra\verseline
que é uma palavra\verseline
nunca usada\verseline
o mesmo nome\verseline
do homem
\end{stanza}
\poemsubtitle{h.4.}
\begin{stanza}
havia também um anjo\verseline
mas só cri por um dia\verseline
voltei ao ateísmo de sempre\verseline
e ao cinismo\verseline
quando de repente\verseline
de repente
\end{stanza}
\poemsubtitle{h.5.}
\begin{stanza}
mas isso que quero\verseline
e temo ter\verseline
essa dança da cor\verseline
de sangue de artéria\verseline
não tem nome de fogo\verseline
nem de gente\verseline
nem de qualquer\verseline
matéria conhecida\verseline
é um algo que não\verseline
o que me queima\verseline
e não se conhece por\verseline
labareda\verseline
é o impossivel\verseline
o zero ao invés\verseline
o tudo
\end{stanza}
\poemsubtitle{h.6.}
\begin{stanza}
mas agora que finalmente falei\verseline
de danças de olhos de tênis\verseline
e tais coisas\verseline
é como se
\end{stanza}
\poemsubtitle{h.6.1.}
\begin{stanza}
mas isso não é.\verseline
simplesmente não.
\end{stanza}
\poemsubtitle{i.}
\begin{stanza}
essa palavra que não quero dizer\verseline
\qquad \sout{DESEJO}\verseline
tem o mesmo nome do medo\verseline
o que é intenso\verseline
um vazio no centro
\end{stanza}
\poemsubtitle{j.}
\begin{stanza}
saber nomes de poetas\verseline
e recitar versos\verseline
de madrugada\verseline
em voz alta\verseline
eu sentia que estava\verseline
cheio de vida\verseline
e por isso\verseline
estava morrendo
\end{stanza}
\poemsubtitle{k.}
\begin{stanza}
caótico disperso\verseline
desorganizado \qquad assim\verseline
o que corre na chuva\verseline
para se sentir vivo\verseline
se estar vivo\verseline
é estar molhado \qquad assim\verseline
com a mente em parafuso\verseline
é uma confusa\verseline
epifaina negativa \qquad déjà vu\verseline
de um déjà vu \qquad quem\verseline
já sonhou\verseline
um sonho dentro de um\verseline
\qquad sabe do que eu falo\verseline
o pavor do inescapável\verseline
\qquad quem conhece a\verseline
\qquad paralisia do sono\verseline
quem lutou com deus\verseline
que o não deixava\verseline
dormir nem acordar\verseline
\qquad quem pensa em\verseline
\qquad arte\verseline
(esta uma é menos cabral)\verseline
essas coisas que saem num vômito\verseline
eu queria não escrever\verseline
como quem mija\verseline
\qquad \qquad \qquad antes\verseline
\qquad \qquad \qquad \qquad eu queria\verseline
viver como quem mija\verseline
não como quem\verseline
pede desculpa
\end{stanza}
\poemsubtitle{l.}
\begin{stanza}
o que eu farei\verseline
quando acabarem as letras?\verseline
al-kawarizmi
\end{stanza}
\poemsubtitle{m.}
\begin{stanza}
mas falamos daquilo\verseline
\qquad \qquad secreto\verseline
mesmo quando revelado\verseline
desejar o objeto\verseline
\qquad de desejo\verseline
e ser dele o\verseline
\qquad de desprezo\verseline
ainda que não o despreze\verseline
e até o queira bem\verseline
o problema aqui é\verseline
a proporção\verseline
uma questão\verseline
de não tanto quanto
\end{stanza}
\poemsubtitle{n.}
\begin{stanza}
ecos de fracassos. \qquad caminhando\verseline
sobre destroços de pretensão\verseline
\qquad o orgulho por um segundo\verseline
\qquad logo em cacos\verseline
vestígios de ter tentado algo\verseline
tentado errado\verseline
\qquad \qquad *\verseline
amar a humilhação e a tortura\verseline
e não tirar nada de bom\verseline
dessas duas \qquad so sentir\verseline
a crueldade\verseline
de quem se esforça em rir\verseline
cotidianamente o desrespeito\verseline
ao que de precioso\verseline
se oferece\verseline
\qquad \qquad *
\end{stanza}
\begin{stanza}
não era disso que se falava
\end{stanza}
\begin{stanza}
é uma pena ter que falar em tempo
\end{stanza}
\begin{stanza}
estar no tempo
\end{stanza}
\begin{stanza}
quando ele não é nada\verseline
mais que uma água
\end{stanza}
\begin{stanza}
um desespero
\end{stanza}
\begin{stanza}
livro escrito em estilo muito pobre\verseline
o anteriormente anunciado
\end{stanza}
\begin{stanza}
são só palavras \qquad depois \qquad antes
\end{stanza}
\begin{stanza}
\qquad \qquad *
\end{stanza}
\begin{stanza}
lembro ainda quando a vida\verseline
era inteira\verseline
feita de merda
\end{stanza}
\poemsubtitle{o.}
\begin{stanza}
mas essa palavra que não quero dizer\verseline
isso que queima e se chama\verseline
chama\verseline
são frases e poemas\verseline
em sua dança\verseline
e a memória de que\verseline
ainda há vida\verseline
mesmo que a morte\verseline
esteja caótia \quad em parafuso\verseline
mesmo sem certeza nenhuma\verseline
é so insegurança\verseline
e medo de ficar\verseline
sozinho para sempre\verseline
e sem ninguém\verseline
e nunca provar\verseline
o que há de bom\verseline
\qquad \qquad \qquad mesmo que \qquad nada\verseline
a grande mentira\verseline
não há possível\verseline
justificativa
\end{stanza}
\poemsubtitle{p.}
\begin{stanza}
não vejo nada\verseline
não vejo a fita\verseline
dominada\verseline
\qquad eu vejo os preto\verseline
\qquad sempre triste\verseline
\qquad nos canto do mundão
\end{stanza}
\attribution{--- Mano Brown}
\poemsubtitle{q.}
\begin{stanza}
que horrível é poder não usar as palavras\verseline
ainda que elas existam!
\end{stanza}
\poemsubtitle{u.}
\begin{stanza}
folha seca num\verseline
vendaval\verseline
um inútil:\verseline
é morrer aos poucos\verseline
eu me sentia assim,\verseline
tio
\end{stanza}
\attribution{--- Mano Brown}
\poemsubtitle{v.}
\begin{stanza}
escrevendo\verseline
a sensação constante\verseline
de estar pondo merda\verseline
no papel
\end{stanza}
\poemsubtitle{w.}
\begin{stanza}
lembra do espírito\verseline
do 14 de julho?\verseline
ele é tão vazio\verseline
quanto ser enrolado\verseline
pelas fantasias
\end{stanza}
\begin{stanza}
as madrugadas\verseline
dedicadas\verseline
ao nada
\end{stanza}
\begin{stanza}
aquilo que se chama\verseline
loucura\verseline
de querer o que não tem
\end{stanza}
\begin{stanza}
e o que\verseline
sabendo disso\verseline
se mantém perto\verseline
sem estar totalmente
\end{stanza}
\begin{stanza}
creio que aqui voltamos\verseline
ao espírito \verseline
do primeiro ensaio\verseline
deste livro
\end{stanza}
\begin{stanza}
o que é menor arte\verseline
que isto?
\end{stanza}
\begin{stanza}
ter dezesseis anos\verseline
sentir-se ridículo\verseline
algo supera?
\end{stanza}
\poemsubtitle{x.}
\begin{stanza}
acabei de inventar\verseline
uma nova fantasia:\verseline
voz aguda\verseline
meu deus\verseline
tenho que trabalhar\verseline
nada me é dado\verseline
e nem permitem\verseline
o suicídio:\verseline
\qquad quando mais eu escrevo\verseline
\qquad menos estou escrevendo\verseline
de natural já não digo nada\verseline
agora nem mais construo\verseline
\quad só estou lembrando\verseline
\qquad \qquad \qquad \qquad O QUÊ?\verseline
\rule{4cm}{0.4pt} como é\verseline
possivel ser poeta\verseline
sendo tão egoísta? \underline{ } \underline{ } \underline{ } \underline{ } \underline{ } \underline{ } \underline{ }
\end{stanza}
\clearpage
\poemsubtitle{y.}
\begin{stanza}
One of these days\verseline
I’m going to cut you\verseline
into little pieces
\end{stanza}
\attribution{--- Pink Floyd}
\poemsubtitle{z.}
\begin{stanza}
é um número z\verseline
simétrico\verseline
no alfabeto\verseline
da letra \textsc{z} (dois) \rule{4.25cm}{0.4pt}\verseline
\rule{4cm}{0.4pt} chega de conversa\verseline
vamos direto\verseline
ao que interessa\verseline
não sei o que é tambem\verseline
mas tudo bem\verseline
vamos\verseline
preciso de força\verseline
pelo menos\verseline
para escrever até o fim deste livro\verseline
\rule{2cm}{0.4pt} não para viver depois disso.
\end{stanza}
\end{poem}

\begin{center}
*\\
*\\
*\\
*\\
*\\
*\\
*
\end{center}