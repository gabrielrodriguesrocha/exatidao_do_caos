\begin{poem}
\sequencetitle{“Tanta lucidez da vertigem”}
\begin{stanza}
Levanta ainda com vestígio\verseline
de absurdo fechando o olho\verseline
precisa uma bebida áspera
para afiar a lâmina\verseline
de ver com lucidez
\end{stanza}
\begin{stanza}
Anda em exercício\verseline
de enxergar as coisas coisas\verseline
não além não menos\verseline
Não se move de si senão para\verseline
branquear o branco\verseline
endurecer o diamante
\end{stanza}
\begin{stanza}
Fala o objeto em gesto reto\verseline
da mão\verseline
que varia o ângulo\verseline
e não faz arco\verseline
Quem sabe a agulha\verseline
de milhões de vértices\verseline
do redondo?\verseline
Do paradoxo umami\verseline
gosto ímpar\verseline
cinco\verseline
mas de aresta polida\verseline
sem excesso sem falta
\end{stanza}
\begin{stanza}
Para quem sai da cama\verseline
com um ritmo irritante\verseline
no ouvido\verseline
nada é mais tortura e desafogo\verseline
que encostar a orelha\verseline
em si mesmo e ouvir\verseline
os vários ecos do oco
\end{stanza}
\begin{stanza}
***
\end{stanza}
\end{poem}