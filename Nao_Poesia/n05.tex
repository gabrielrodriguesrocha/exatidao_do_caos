\begingroup
\huge
\justify
O labirinto a ser criado é o labiritno da vertigem da linguagem. O labirinto da lucidez da palavra coisa. Com seus caminhos que se bifurcam. Por exemplo:$\left\{\begin{tabular}{c}uma coisa \\ outra coisa\end{tabular}\right .$, e nem sempre simétricos em espelho. E sempre a libertação do já feito e do alheio. Ainda que eu queira ser um eco ou continuador. É ainda em oposição que se faz essa continuação.
\begin{center}{\rule{4cm}{0.4pt}\raisebox{-0.5ex}{x}\rule{4cm}{0.4pt}}\end{center}
\begin{justify}
Oposição a tudo, sem matar nada nem ninguém. Mas tambem, trata-se aqui do so called sistema poético estabelecido de antemão, que é \underline{meu} sistema, não regras ditadas a ninguém. Regras so pra mim mesmo. O Tarik disse: estudar. Estudar até que se me posso talvez chamar poeta, ou escritor.
\end{justify}
\endgroup