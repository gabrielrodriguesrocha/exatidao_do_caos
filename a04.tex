\begin{poem}
\sequencetitle{4. definições}
\poemsectiontitle{1.}
\begin{stanza}
arte é o vazio refletido no espelho\verseline
enxergar através de lentes\verseline
antimiopemente \qquad fazer questão\verseline
de que a água seja bem peneirada\verseline
correr atrás do vento \qquad pra usar\verseline
uma referência clássica\verseline
é fazer com que o velho\verseline
pareça nascido agora\verseline
e reformar os olhos com catarata\verseline
reciclar os ouvidos dos surdos\verseline
deformar o que está aí\verseline
para que mudando tudo\verseline
se chegue à forma real das coisas
\end{stanza}
\poemsectiontitle{2.}
\begin{stanza}
ela disse\verseline
\qquad Tudo é arte\verseline
\qquad \qquad e eu ia começar a dizer\verseline
\qquad Não…\verseline
– ela me interrompeu com aquele olhar\verseline
que significa\verseline
\qquad Já vem você me chamar de burra
\end{stanza}
\begin{stanza}
\qquad (como se eu não fosse a carne\verseline
\qquad que diz sim pra tudo\verseline
\qquad aquele que é ofendido\verseline
\qquad e pede desculpas\verseline
\qquad o desprezível desprezado\verseline
\qquad que se humilha se rebaixa\verseline
\qquad para que os outros sejam\verseline
\qquad os glorificados\verseline
\qquad escondido debaixo\verseline
\qquad das solas dos príncipes\verseline
\qquad do mundo\verseline
\qquad esses outros que nunca\verseline
\qquad jamais levam porrada\verseline
\qquad os que apontam e riem\verseline
\qquad dos que só têm de seu\verseline
\qquad coisas emprestadas\verseline
\qquad – usadores de palavras)
\end{stanza}
\poemsectiontitle{3.}
\begin{stanza}
inútil dizer o que é o poema
\end{stanza}
\begin{stanza}
\qquad o poema é esse fazer e refazer o nada\verseline
\qquad \qquad \qquad \qquad do nada – 
\end{stanza}
\begin{stanza}
silêncios exaltados
\end{stanza}
\begin{stanza}
nem poucas nem mais palavras
\end{stanza}
\begin{stanza}
palavra.
\end{stanza}
\poemsectiontitle{4.}
\begin{stanza}
precisamos de algo mais que definições\verseline
precisamos de edificações de areia\verseline
de fortificações de ar\verseline
\qquad \qquad precisamos de sonhos antes de tudo\verseline
\qquad sonhos para realizar dar vender\verseline
ou enterrar no quintal de casa\verseline
\qquad os meus ideais estão num lugar bem seguro\verseline
enfiados onde ninguém vai pôr a mão\verseline
\qquad a minha segurança são os cadeados\verseline
e os cadeados dos cadeados\verseline
\qquad a minha segurança\verseline
é que hoje tudo é automático\verseline
(falo hoje como se houvesse o passado)\verseline
e todos podem sair sabendo que ao voltar\verseline
seus segredos estarão bem guardados\verseline
na boca dos amigos dos amigos dos amigos\verseline
dos conhecidos dos amigos dos conhecidos\verseline
dos ex e dos ex dos ex amigos\verseline
e nem digo nas bocas digo nos dedos\verseline
digo nas redes digo nos bytes dos sites\verseline
\qquad lugares onde a eternidade\verseline
\qquad é transitória
\end{stanza}
\end{poem}