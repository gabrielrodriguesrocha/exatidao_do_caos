\begin{poem}
\poemsubtitle{1.}
\begin{stanza}
encontro consigo\verseline
cego luz do\verseline
espelho bipartido\verseline
de fim a fim
\end{stanza}
\begin{stanza}
essa coisa humana\verseline
mente se expondo\verseline
em dois gumes na cara\verseline
da cara \quad espécie\verseline
de objetos jamais\verseline
com utilidade
\end{stanza}
\begin{stanza}
e por acaso corre\verseline
as mãos na pele\verseline
e acha\verseline
aquela cicatriz\verseline
um souvenir?\verseline
esquecido\verseline
que era assim doce
\end{stanza}
\begin{stanza}
assim como\verseline
sem alguma coerência\verseline
se passa do sujeito\verseline
para a consequência\verseline
mordente nesse caso
\end{stanza}
\begin{stanza}
de repente blue devils\verseline
pensa \quad e quem sabe\verseline
uma bala na ideia\verseline
não impediria\verseline
a cabeça
\end{stanza}
\poemsubtitle{2.}
\begin{stanza}
entretanto\verseline
o quase é sempre\verseline
e tudo que existe
\end{stanza}
\begin{stanza}
o quase é mais\verseline
vazio que o vazio
\end{stanza}
\begin{stanza}
exemplo
\end{stanza}
\begin{stanza}
correr no mesmo\verseline
passo\verseline
atrás de quem\verseline
se move um passo\verseline
de cada vez
\end{stanza}
\begin{stanza}
exemplo
\end{stanza}
\begin{stanza}
estar a um paço\verseline
só de poço\verseline
ou precipício
\end{stanza}
\begin{stanza}
exemplo
\end{stanza}
\begin{stanza}
\underline{sempre} a um passo\verseline
são outros quinhentos\verseline
quilômetros entre\verseline
quem e quem
\end{stanza}
\end{poem}