\center{
Epígrafes, Prefácio, Prelúdio ou\\
PRENÚNCIO PROFÉTICO DO QUE HÁ DE VIR\\}
\begin{poem}
\sequencetitle{DIANTE DA FOLHA BRANCA}
\begin{stanza}
Tanta lucidez da vertigem.\marginpar{\small{Van Gogh}}\verseline
Faz perder o pé na realidade.\verseline
Perder pé dentro de si mesmo,\verseline
sem contrapé, é uma voragem.
\end{stanza}
\begin{stanza}
Diante da folha branca e virgem,\verseline
na mesa, e de todo ofertada,\verseline
com medo de que ela sorvesse,\verseline
ei-lo, como louco, a estuprá-la.
\end{stanza}
\begin{stanza}
\hspace{2cm}*
\end{stanza}
\begin{stanza}
A folha branca é a tradução\marginpar{\small{Mallarmé}}\verseline
mais aproximada do nada.\verseline
Por que romper essa pureza\verseline
com palavra não nilpesada?
\end{stanza}
\begin{stanza}
A folha branca não aceita\verseline
senão a que acha que a merece:\verseline
essa so sobrevive ao fogo\verseline
desse branco que é gelo e febre.
\end{stanza}
\attribution{João Cabral de Melo Neto, in Agrestes (1981-1985)}
\end{poem}