\begin{poem}
\sequencetitle{2. voyeur}
\poemsectiontitle{a.}
\begin{stanza}
suéter de losangos e óculos de coração\verseline
(alguma lolita com frio)\verseline
esperava o ônibus perto da rua\verseline
e sorriu \qquad juro por deus\verseline
de mostrar os dentes\verseline
por trás do batom vermelho
\end{stanza}
\begin{stanza}
me entristece perceber que eu descrevi a mulher\verseline
como quem descreve uma caixa de frutas\verseline
ou um copo com um resto de leite\verseline
em cima da mesa\verseline
me perco nessas voltas\verseline
mas nem aprendi\verseline
a usar as palavras)
\end{stanza}
\poemsectiontitle{b.}
\begin{stanza}
inutilmente esperei um milagre\verseline
de pé encostado na grade\verseline
enquanto o ônibus não vinha\verseline
ninguém se jogou no meu pescoço\verseline
ninguém ninguém nos meus braços
\end{stanza}
a ficção alimenta sonhos falsos\verseline
mas alimenta sonhos
\begin{stanza}
essas meninas têm o rosto impermeável\verseline
maquiagem a prova d'água e de teorias\verseline
antes da viagem \qquad antes de tudo\verseline
a poesia não tem a menor impotência\verseline
o poeta grita no livro fechado\verseline
mas além de livros um país\verseline
se faz de homens e mulheres\verseline
de mulheres e mulheres\verseline
de homens e homens\verseline
de palavras\verseline
de ideias\verseline
etc.
\end{stanza}
\poemsectiontitle{c.}
\begin{stanza}
dentro de si é uma mala 007\verseline
de que ninguém sabe o segredo\verseline
(exceto é claro aquele amigo\verseline
matemático mestre em combinatória\verseline
e convívio social)
\end{stanza}
\begin{stanza}
maleta dessas que se viola a tiro\verseline
mas eu não saio abrindo\verseline
os mistérios de ninguém\verseline
por muito menos já morri\verseline
por muito menos outros\verseline
já perderam o ponto
\end{stanza}
\begin{stanza}
uma pessoa que na vida\verseline
só chegou atrasada\verseline
por trocar o sim pelo não e vice-versa
\end{stanza}
\begin{stanza}
ando a pé
não corro o risco de ficar muito tempo\verseline
me prender a quem seja na calçada\verseline
e esquecer \qquad a verdade
\end{stanza}
\begin{stanza}
\center{a verdade \qquad a verdade \qquad a verdade}\verseline
\center{\textit{[[repeat]]}}
\end{stanza}
\end{poem}