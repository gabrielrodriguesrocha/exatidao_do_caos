\documentclass[20pt]{book}
\usepackage{fancyhdr,poemscol}
\usepackage[utf8]{inputenc}
\usepackage[margin=1.5in]{geometry}
\usepackage[brazilian]{babel}

\title{A EXATIDÃO DO CAOS}
\author{ANDRÉ MIRANDA SILVA}
\date{}
\global\verselinenumbersfalse

\begin{document}
\maketitle

\newpage
\thispagestyle{empty}
\mbox{}
\newpage

\par\null\vfill
\begin{center}\Large{ASTENIA}\end{center}
\begin{flushright}
\textit{
as.te.ni.a s.f. MED perda ou diminuição\\
da força física”\\
– Houaiss}
\end{flushright}
\clearpage
ASTENIA é essa eterna saída do caminho no meio dele. É esse
sempre parar com as coisas sem que nada tenha acontecido. É
encontrar muros e não subi-los. É ser impedido e unca impelido. É
sair do jogo sem começar a jogar. É a impotência, a impossibilidade
de encarar o desafio. É a perplexidade diante das coisas do mundo. A
incapacidade de apreender as informações e processá-las. É não
entender e não ser entendido. É sempre o equívoco e a insegurança
depois. Astenia é não viver.
\clearpage
\begin{poem}
\begin{stanza}
eu sei eu sei\verseline
que tem muito\verseline
que eu não sei\verseline
\qquad o que é melhor\verseline
\qquad olhar a lua cheia\verseline
\qquad ou fotografias\verseline
\qquad de abelhas?\verseline
o mundo bate\verseline
ou melhor\verseline
as coisas\verseline
\qquad eu não tô na reali\verseline
\qquad são muitos caminhos\verseline
\qquad e muitas escolhas\verseline
ou talvez\verseline
tudo é tão simples\verseline
que eu ainda não vi\verseline
\qquad não sei não sei\verseline
\qquad nem vivi
\end{stanza}
\begin{stanza}
\quad cansei.
\end{stanza}
\end{poem}
\clearpage
\begin{poem}
\begin{stanza}
mas enquanto descrevem a\verseline
estrutura do dia ou\verseline
a gramática da língua\verseline
em folhas de papel\verseline
\qquad folhas caem\verseline
\qquad das árvores\verseline
\qquad e não podem\verseline
\qquad ser faladas\verseline
as palavras são muito bonitas\verseline
mas \quad já não bastam
\end{stanza}
\end{poem}
\clearpage
\begin{poem}
\begin{stanza}
se ninguém viu \qquad ouviu\verseline
não existe\verseline
\qquad as folhas dos livros\verseline
\qquad as folhas das árvores\verseline
\qquad têm o mesmo valor\verseline
\qquad e são inúteis\verseline
se o sol explodisse\verseline
eu morreria\verseline
agora ou\verseline
em bilhões de anos\verseline
\qquad se eu morrer eu não tenho\verseline
\qquad \qquad nenhum plano b
 \end{stanza}
 \end{poem}
\clearpage
\begin{poem}
\begin{stanza}
preciso que me dê a mão\verseline
porque a vida é infinita\verseline
e o instante me assusta\verseline
\qquad ou todo mundo é ruim\verseline
\qquad ou sou só eu\verseline
que estou lá fora\verseline
ao meio dia\verseline
hora em que nada\verseline
nada circula\verseline
\qquad se eu dissesse\verseline
\qquad uma mentira\verseline
\qquad sei que não gostaria\verseline
a vida são só pastos\verseline
e pontos de ônibus\verseline
\qquad por isso vamos\verseline
\qquad correr correr\verseline
\qquad até cair\verseline
mesmo sem chegar\verseline
a nenhum lugar nenhum\verseline
\qquad o importante\verseline
\qquad é estar.
\end{stanza}
\end{poem}
\clearpage
\begin{poem}
\begin{stanza}
fala\verseline
é mais fácil\verseline
que a desfala\verseline
e mais difícil\verseline
que a não fala
\end{stanza}
\begin{stanza}
resvala no\verseline
querer dito\verseline
o que se\verseline
quer dizer
\end{stanza}
\begin{stanza}
e o que\verseline
se quer\verseline
se deve\verseline
nunca\verseline
interditar
\end{stanza}
\begin{stanza}
raiva\verseline
contra o mecanismo\verseline
um abissal\verseline
desejo\verseline
de um corpo\verseline
ou copo\verseline
d'água
\end{stanza}
\begin{stanza}
o abismo\verseline
da não fala\verseline
é vazio\verseline
e muitos são\verseline
os que o acham
\end{stanza}
\begin{stanza}
(sai dele\verseline
\ sai dele\verseline
\ meu povo)
\end{stanza}
\begin{stanza}
os que não\verseline
podem\verseline
o trabalho\verseline
de se abrir
\end{stanza}
\begin{stanza}
os que não\verseline
querem\verseline
o peso\verseline
de sorrir
\end{stanza}
\end{poem}
\clearpage
\begin{poem}
\begin{stanza}
As palavras são muito bonitas\verseline
mas já não bastam\verseline
\qquad a vida é vivida\verseline
\qquad em não viver\verseline
vidinha de rotinas\verseline
e conveniências\verseline
\qquad NINGUÉM MAIS GRITA\verseline
o poeta que vive em mim\verseline
dorme o dia inteiro\verseline
\qquad o deus que vive em mim\verseline
\qquad se desconverteu\verseline
nem sempre se pode ter\verseline
tudo que se quer\verseline
\qquad as cordas vibram\verseline
\qquad o baixo sobressai\verseline
\qquad a bateria explode\verseline
\qquad \qquad mas\verseline
ninguém grita mais
\end{stanza}
\end{poem}
\clearpage
\begin{poem}
\begin{stanza}
se me pegam na rua\verseline
já era eu\verseline
\qquad a boca se cala\verseline
\qquad e nisso acerta\verseline
a língua é muito falha\verseline
e não serve\verseline
pra defender\verseline
\qquad nem posso me esconder\verseline
\qquad por trás caneta
\end{stanza}
\end{poem}
\clearpage
\begin{poem}
\begin{stanza}
recebo\verseline
\qquad um olhar\verseline
\qquad um sorriso\verseline
e ando mais rápido\verseline
\qquad desvio
\end{stanza}
\begin{stanza}
medo\verseline
\qquad de realizar
\end{stanza}
\end{poem}
\clearpage
\begin{poem}
\begin{stanza}
se é pelo bem de todos\verseline
e a felicidade geral da nação\verseline
– mente sã\verseline
\quad corpo são –\verseline
\qquad passaremos sim\verseline
\qquad em frente a vocês\verseline
\qquad na primeira do plural\verseline
passaremos sim\verseline
de mãos dadas\verseline
de braços dados\verseline
de corpos grudados\verseline
\qquad passaremos sim\verseline
\qquad com fones de olvido\verseline
\qquad para esquecer com que frequência\verseline
\qquad vocês nos iludem\verseline
passaremos sim\verseline
de peitos abertos\verseline
e olhos lavados\verseline
para ver o fim\verseline
e o que vem\verseline
muito antes do começo\verseline
\qquad passaremos sim\verseline
\qquad todos juntos\verseline
\qquad para verem o quanto somos\verseline
\qquad diferentes e iguais\verseline
passaremos sim\verseline
ao vivo ao meio dia\verseline
para que vocês não precisem\verseline
nem ver seus jornais\verseline
\qquad passaremos sim\verseline
\qquad para que vejam\verseline
\qquad o quanto odiamos\verseline
\qquad uns aos outros\verseline
mas para que vejam\verseline
que entre nós\verseline
há sempre a certeza\verseline
do perdão
\end{stanza}
\end{poem}
\clearpage
\begin{poem}
\begin{stanza}
do alto das sarjetas\verseline
do fundo das calçadas\verseline
de dentro das carteiras\verseline
debaixo das janelas\verseline
\qquad silêncios\verseline
\qquad condições\verseline
\qquad cautelas\verseline
desde o passado até agora\verseline
reclamar não deu em nada\verseline
não é enredo de novela\verseline
os fracos não têm vez\verseline
se você fez algo\verseline
ninguém sabe que fez\verseline
é preciso mais que vida\verseline
mais que funções vitais\verseline
é preciso mais que arte\verseline
é preciso superar\verseline
\qquad vejo arte\verseline
\qquad em toda parte\verseline
\qquad vejo lixo\verseline
\qquad já vi morte\verseline
mas eu não vi\verseline
a solução\verseline
\qquad \qquad que é a força\verseline
\qquad \qquad \qquad que é a firme\verseline
\qquad \qquad resolução
\end{stanza}
\end{poem}
\clearpage
\begin{poem}
é apenas o começo\verseline
andamos calados\verseline
criando aqui dentro\verseline
uma desobediência\verseline
\qquad mas quem somos nós?\verseline
\qquad \qquad um pronome\verseline
\qquad \qquad rostos frágeis\verseline
a força que calamos\verseline
não vai explodir?\verseline
\hspace{5em}(que força?)\verseline
\qquad poderia ser\verseline
\qquad no papel\verseline
\qquad na tela\verseline
\qquad ou na voz\verseline
\hspace{5em} digital\verseline
mas as palavras\verseline
elas não bastam\verseline
\qquad não
\end{poem}
\clearpage
\hspace{4em}Poema artefato \textit{vs}. Poema discurso. Sem versus. Só versos. \\
\indent\hspace{4em}(Sem trocadilhos ruins como esse). Construir aquela rua de
que eu falei. De dentro pra fora. Montar a crocância da casca do pão. \\
\indent\hspace{4em}Emoção? Emoção. Construção. Como criar e ainda mostrar o
que está? Sem dilemas…(?) Sem tensões…(?) \\
\indent\hspace{12em}Poesia é problema (?)
\clearpage
\noindent Mas não posso entender que algo nasça do não. A fria negação. A luz
trêmula e pálida. Nada serve como alimento. As coisas têm de ser
vistas nas suas devidas proporções. Mesmo que assim seja: não ir à
festa. Rato de biblioteca. Fichas de cartolina. Mesmo que não seja
assim: diversão é solução, sim. É desse modo que se vive – através
dele. E seu muito poder. Sim, senhor. Através do sim.
\clearpage
\noindent As coisas escapam por entre os dedos. O mundo. O que acabou de
acontecer. Como andar de olhos fechados. O tempo é uma poeira
fininha. Inalcançável. Inatingível. Intangível. Todos os adjetivos.
Principalmente na rotina so-nâm-bu-la. A retina tão dificilmente
excitável. Olhar olhar e não ver nada. Viver sem saber. O mundo
foge. As coisas. Viver sem viver. Sem viver. Astenia.
\center{\textit{28.4.2015}}
\clearpage
\begin{poem}
\begin{stanza}
o sol refletidono concreto\verseline
machuca as rotinas\verseline
no passo sonâmbulo dos dias\verseline
as calçadas escorrem\verseline
debaixo dos pés que buscam\verseline
uma orfandade voluntária\verseline
aquela outra sozinhez\verseline
que é dentro de si\verseline
em meio a outros
\end{stanza}
\begin{stanza}
\qquad CONSERTA-SE:\verseline
\qquad celulares – tablets – PCs\verseline
e nada mais
\end{stanza}
\begin{stanza}
ninguém pode imaginar a paz\verseline
contida em um copo d'água
\end{stanza}
\end{poem}
\clearpage
\begin{poem}
\begin{stanza}
o ser transtornado\verseline
se devora em perguntas\verseline
surdas ao extremo:\verseline
o aqui dentro? o lá fora?\verseline
até onde chegar\verseline
à força de verdades?\verseline
e não é sua lembrança\verseline
que passa na janela\verseline
através de muitos metros\verseline
na neblina?\verseline
um segundo passageiro\verseline
se levanta\verseline
cansado de pensamentos repetidos\verseline
até o absurdo.\verseline
o tempo parado\verseline
absoluto\verseline
impede que as coisas\verseline
se dissolvam.\verseline
nem tudo se resolve\verseline
com falagens\verseline
(o general e sua\verseline
falange imperial\verseline
beberam o sangue\verseline
do inimigo –\verseline
beberam um pôr-de-sol\verseline
terra vermelha\verseline
vaso de argila)\verseline
as pontas soltas do passado\verseline
levantadas pelo vento\verseline
marcando os umbrais\verseline
das portas abertas do presente:\verseline
passaremos? ficaremos?\verseline
são dúvidas duplicadas\verseline
na lâmina dura da água\verseline
à beira da estrada
\end{stanza}
\end{poem}
\clearpage

\par\null\vfill
\begin{center}
\Large{
LONG PLAY\\
(365 rpm)}
\end{center}

\clearpage

\par\null\vfill
\begin{center}
\Large{
LADO A}
\end{center}
\begin{flushright}
\large{
\textit{- “Deem-me uma outra vida e estarei cantando...”\\
- Iósif Bródski}, Para minha filha
}
\end{flushright}

\begin{poem}
\sequencetitle{1. Intro}
\begin{stanza}
\qquad mal nasci\verseline
\qquad já planejo crimes\verseline
 – que eu traia\verseline
mas não seja nunca traído\verseline
por esta palavra:\verseline
\qquad (ou esta:\verseline
\qquad \qquad ou esta:\verseline
\qquad \qquad \qquad ou esta:)\verseline
 – que eu invada\verseline
mas não seja nunca\verseline
invadido por este pudor\verseline
\qquad \qquad este desejo escondido\verseline
\qquad \qquad de não viver\verseline
\qquad \qquad de sentar em cantos\verseline
\qquad \qquad \qquad de paredes\verseline
\qquad \qquad e responder o eco\verseline
\qquad \qquad \qquad da própria voz\verseline
\qquad – que eu tema\verseline
\qquad mas não seja nunca vítima\verseline
\qquad do medo dos outros\verseline
\qquad que tem calado a voz\verseline
\qquad dos nossos abraços\verseline
 – que eu roube\verseline
mas que nunca tirem de mim\verseline
o que eu tenho de eterno:\verseline
\quad as paredes do instante\verseline
\quad que bloqueiam\verseline
\quad que são maiores que os antes
\end{stanza}
\end{poem}
\clearpage
\begin{poem}
\sequencetitle{2. voyeur}
\poemsectiontitle{a.}
\begin{stanza}
suéter de losangos e óculos de coração\verseline
(alguma lolita com frio)\verseline
esperava o ônibus perto da rua\verseline
e sorriu \qquad juro por deus\verseline
de mostrar os dentes\verseline
por trás do batom vermelho
\end{stanza}
\begin{stanza}
me entristece perceber que eu descrevi a mulher\verseline
como quem descreve uma caixa de frutas\verseline
ou um copo com um resto de leite\verseline
em cima da mesa\verseline
me perco nessas voltas\verseline
mas nem aprendi\verseline
a usar as palavras)
\end{stanza}
\poemsectiontitle{b.}
\begin{stanza}
inutilmente esperei um milagre\verseline
de pé encostado na grade\verseline
enquanto o ônibus não vinha\verseline
ninguém se jogou no meu pescoço\verseline
ninguém ninguém nos meus braços
\end{stanza}
a ficção alimenta sonhos falsos\verseline
mas alimenta sonhos
\begin{stanza}
essas meninas têm o rosto impermeável\verseline
maquiagem a prova d'água e de teorias\verseline
antes da viagem \qquad antes de tudo\verseline
a poesia não tem a menor impotência\verseline
o poeta grita no livro fechado\verseline
mas além de livros um país\verseline
se faz de homens e mulheres\verseline
de mulheres e mulheres\verseline
de homens e homens\verseline
de palavras\verseline
de ideias\verseline
etc.
\end{stanza}
\poemsectiontitle{c.}
\begin{stanza}
dentro de si é uma mala 007\verseline
de que ninguém sabe o segredo\verseline
(exceto é claro aquele amigo\verseline
matemático mestre em combinatória\verseline
e convívio social)
\end{stanza}
\begin{stanza}
maleta dessas que se viola a tiro\verseline
mas eu não saio abrindo\verseline
os mistérios de ninguém\verseline
por muito menos já morri\verseline
por muito menos outros\verseline
já perderam o ponto
\end{stanza}
\begin{stanza}
uma pessoa que na vida\verseline
só chegou atrasada\verseline
por trocar o sim pelo não e vice-versa
\end{stanza}
\begin{stanza}
ando a pé
não corro o risco de ficar muito tempo\verseline
me prender a quem seja na calçada\verseline
e esquecer \qquad a verdade
\end{stanza}
\begin{stanza}
\center{a verdade \qquad a verdade \qquad a verdade}\verseline
\center{\textit{[[repeat]]}}
\end{stanza}
\end{poem}
\clearpage
\begin{poem}
\sequencetitle{3. objeto de desejo}
\begin{stanza}
enquanto você não está aqui\verseline
rugindo suas músicas de\verseline
pré-duplo-homicídio-suicídio\verseline
remastigando essas lembranças\verseline
remasterizadas\verseline
de um passado inútil\verseline
ou a nostalgia futura\verseline
de um tempo impossível\verseline
e descafeinado \qquad essas\verseline
fantasias que se vê em toda esquina
\end{stanza}
\begin{stanza}
enquanto você não está se lamentando\verseline
e eu não me lamento pra você\verseline
de não ter agarrado enquanto podia\verseline
todas as chances que o mundo\verseline
dava \quad dava \quad voltas e eu imóvel\verseline
bem como um móvel na sala\verseline
um sofá \qquad calado e útil\verseline
(você me chamava de\verseline
\qquad \qquad criado mudo\verseline
e eu não sabia o que era isso)
\end{stanza}
\begin{stanza}
enquanto você não está fazendo\verseline
seu habitual espetáculo\verseline
\qquad (a vida é um cinema em\verseline
\qquad \qquad dia de chuva)\verseline
ou torturando as pessoas com\verseline
sua voz de navalha na carne\verseline
ou fazendo ligações perigosas\verseline
\qquad uma vida-montanha-russa\verseline
ou contando vantagem e\verseline
histórias comoventes que mais parecem\verseline
piadas sem graça\verseline
ou servindo nossos olhares\verseline
de mais um exemplar\verseline
da sua antiarte inútil\verseline
ou dizendo futilidades\verseline
da sua prima ou daquela\verseline
sua amiga que bem que\verseline
podia ter morrido\verseline
ou do seu gato\verseline
que é só um pedaço gordo\verseline
de carne de cadáver
\end{stanza}
\begin{stanza}
enquanto você não vem\verseline
é como se eu fosse um estrangeiro\verseline
na minha própria vida
\end{stanza}
\end{poem}
\clearpage
\begin{poem}
\sequencetitle{4. definições}
\poemsectiontitle{1.}
\begin{stanza}
arte é o vazio refletido no espelho\verseline
enxergar através de lentes\verseline
antimiopemente \qquad fazer questão\verseline
de que a água seja bem peneirada\verseline
correr atrás do vento \qquad pra usar\verseline
uma referência clássica\verseline
é fazer com que o velho\verseline
pareça nascido agora\verseline
e reformar os olhos com catarata\verseline
reciclar os ouvidos dos surdos\verseline
deformar o que está aí\verseline
para que mudando tudo\verseline
se chegue à forma real das coisas
\end{stanza}
\poemsectiontitle{2.}
\begin{stanza}
ela disse\verseline
\qquad Tudo é arte\verseline
\qquad \qquad e eu ia começar a dizer\verseline
\qquad Não…\verseline
– ela me interrompeu com aquele olhar\verseline
que significa\verseline
\qquad Já vem você me chamar de burra
\end{stanza}
\begin{stanza}
\qquad (como se eu não fosse a carne\verseline
\qquad que diz sim pra tudo\verseline
\qquad aquele que é ofendido\verseline
\qquad e pede desculpas\verseline
\qquad o desprezível desprezado\verseline
\qquad que se humilha se rebaixa\verseline
\qquad para que os outros sejam\verseline
\qquad os glorificados\verseline
\qquad escondido debaixo\verseline
\qquad das solas dos príncipes\verseline
\qquad do mundo\verseline
\qquad esses outros que nunca\verseline
\qquad jamais levam porrada\verseline
\qquad os que apontam e riem\verseline
\qquad dos que só têm de seu\verseline
\qquad coisas emprestadas\verseline
\qquad – usadores de palavras)
\end{stanza}
\poemsectiontitle{3.}
\begin{stanza}
inútil dizer o que é o poema
\end{stanza}
\begin{stanza}
\qquad o poema é esse fazer e refazer o nada\verseline
\qquad \qquad \qquad \qquad do nada – 
\end{stanza}
\begin{stanza}
silêncios exaltados
\end{stanza}
\begin{stanza}
nem poucas nem mais palavras
\end{stanza}
\begin{stanza}
palavra.
\end{stanza}
\poemsectiontitle{4.}
\begin{stanza}
precisamos de algo mais que definições\verseline
precisamos de edificações de areia\verseline
de fortificações de ar\verseline
\qquad \qquad precisamos de sonhos antes de tudo\verseline
\qquad sonhos para realizar dar vender\verseline
ou enterrar no quintal de casa\verseline
\qquad os meus ideais estão num lugar bem seguro\verseline
enfiados onde ninguém vai pôr a mão\verseline
\qquad a minha segurança são os cadeados\verseline
e os cadeados dos cadeados\verseline
\qquad a minha segurança\verseline
é que hoje tudo é automático\verseline
(falo hoje como se houvesse o passado)\verseline
e todos podem sair sabendo que ao voltar\verseline
seus segredos estarão bem guardados\verseline
na boca dos amigos dos amigos dos amigos\verseline
dos conhecidos dos amigos dos conhecidos\verseline
dos ex e dos ex dos ex amigos\verseline
e nem digo nas bocas digo nos dedos\verseline
digo nas redes digo nos bytes dos sites\verseline
\qquad lugares onde a eternidade\verseline
\qquad é transitória
\end{stanza}
\end{poem}
\clearpage
\begin{poem}
\sequencetitle{5. Notas}
\poemsectiontitle{\textit{\underline{Ontem}}}
\begin{stanza}
Ouço os barulhos aí de fora e sofro. Ai.\verseline
Não adianta olhar pela janela que não vem ninguém.\verseline
Pensei que os diamantes fossem para sempre. Estava enganado.\verseline
Parece que eles mofam e apodrecem quando na sombra da\verseline
verdade jogada na cara.\verseline
Não adianta.\verseline
Não vem ninguém.
\end{stanza}
\poemsectiontitle{\textit{\underline{Há dez dias}}}
\begin{stanza}
A alegria do pão de milho contra as lâminas do álcool.\verseline
Sabor de cobre e fumaça.\verseline
Prevejo que vai começar tudo de novo.\verseline
Estamos preparados para a necessária fuga.\verseline
Imploro a Deus que seja mentira. Me ouviria?\verseline
Imploro que a verdade seja o sonho que eu tive ontem. Pai!\verseline
Como eu sofro!\verseline
Desenhei no chão com giz.\verseline
Um jogo. Um zigue-zague. Contra o tique-taque dos que me\verseline
compram e vendem. Absoluto. Frente a frente não sei falar. Só\verseline
abraços. Um absurdo.\verseline
Desse jeito que nos desespera. Dizes pera. Sinto a aflição de\verseline
seus olhos tão modernos. O que eles querem é o contrário do\verseline
que eu. Por isso sinto esta como que faca de açúcar quando\verseline
estou feliz contigo mas a felicidade não é completa porque por\verseline
mais que eu te toque e ouça você ainda fala uma língua outra.\verseline
Escrevo pra você sob uma rajada de silêncios, emoções\verseline
contrárias. Nunca lerá.\verseline
Mas eu insisto em ver flores e abelhas e lembrar.\verseline
Além disso o modo como você me faz sofrer e flutuar é \verseline
totalmente útil pra essa arte fútil.\verseline
E fatalmente não teremos nenhuma paz.\verseline
Nem rimas.\verseline
Querido diário.
\end{stanza}
\poemsectiontitle{\textit{\underline{Há vince e cinco dias}}}
\begin{stanza}
A última semana.\verseline
Sempre éramos idiotas antes de hoje.\verseline
Ou somos todos os dias mas o fato de ser hoje nos torna cegos\verseline
a essa idiotice.\verseline
O ano começa a acabar.\verseline
Tive que fazer essa tentativa. Se não der não deu e fazer o quê\verseline
\qquad seguir em frente ou em outra direção de modo a nem\verseline
sequer reste um vestígio dessa coisa absurda que se chama.
\end{stanza}
\begin{stanza}
\qquad Vale a pena ler o último volume?\verseline
\qquad Estou pensando em dar uma volta.\verseline
\qquad Ar.\verseline
\textit{(Ontem assisti a um filminho de adolescentes. Ilusões vãs.\verseline
Ricos e bonitos. Transgressão convencional. Vale nada)}.
\end{stanza}
\poemsectiontitle{\textit{\underline{Há trinta e um dias}}}
\begin{stanza}
Essa música me faz sentir insuportavelmente adolescente.\verseline
Insuportável \quad In-su-por-ta-vel-men-te.\verseline
O advérbio e-nor-me-men-te po-lis-sí-la-bo.\verseline
Nem de erva nem de solidão \qquad louco de som.\verseline
Menos lúcido que nunca.
\end{stanza}
\poemsectiontitle{\textit{\underline{Depois de amanhã}}}
\qquad All you need is love\verseline
\qquad \quad \  and all I need\verseline
\qquad \qquad \, \, is you\verseline
\qquad \qquad \quad \,    $<$3
\end{poem}
\clearpage
\begin{poem}
\sequencetitle{6. Notas 2}
\poemsectiontitle{a.}
\begin{stanza}
Essa insuficiência que eu sinto\verseline
essa proibição\verseline
direito negado\verseline
será algum\verseline
resto de passado?
\end{stanza}
\poemsectiontitle{b.}
\begin{stanza}
O silêncio é difícil\verseline
de apreender\verseline
As palavras para ele\verseline
são poucas
\end{stanza}
\poemsectiontitle{c.}
\begin{stanza}
Aprendo a viver\verseline
com lápis e borracha\verseline
e nunca mais\verseline
com a caneta definitiva
\end{stanza}
\end{poem}

\par\null
\qquad - \textit{fevereiro/março}
\clearpage
\begin{poem}
\sequencetitle{7. Encantado}
\begin{stanza}
\qquad pegar\verseline
\qquad \qquad um\verseline
\qquad \qquad \qquad atalho\verseline
\qquad para onde
\end{stanza}
\begin{stanza}
\qquad \qquad os sonhos\verseline
\qquad são poluções noturnas
\end{stanza}
\begin{stanza}
\qquad \qquad ou melhor\verseline
\textit{quando a manhã vem} com aquele\verseline
\qquad sorriso besta\verseline
\qquad \qquad você\verseline
\qquad \quad pega na mão dela\verseline
\qquad e vai\verseline
\qquad \qquad dar um passeio muito chato\verseline
\qquad \qquad \quad mas pode\verseline
\qquad porque sofrer é bom\verseline
\qquad \quad quando o sorriso\verseline
\qquad \qquad \, é bonito
\end{stanza}
\end{poem}
\clearpage
\begin{poem}
\sequencetitle{8. Maquinaria}
\begin{stanza}
planejar essas mentiras\verseline
com a perfeição\verseline
do possível
\end{stanza}
\end{poem}

\par\null
 - 21.4.2015
\clearpage
\begin{poem}
\sequencetitle{9. Choro}
\begin{stanza}
segunda-feira chuvosa e febril\verseline
\qquad e míope\verseline
\qquad e míope\verseline
\qquad e míope
\end{stanza}
\begin{stanza}
(fazer disso um drama)
\end{stanza}
\end{poem}
\clearpage
\begin{poem}
\sequencetitle{10. Notas 3}
\begin{stanza}
\qquad *
\end{stanza}
\begin{stanza}
Somos incapazes de perceber\verseline
que estamos aplaudindo\verseline
um ser abjeto?
\end{stanza}
\begin{stanza}
O que eu fiz em todo esse tempo não significa nada. Esse
tempo vazio. Intermezzo. Essa idade média da minha vida.
Essa nulidade. Amasso o papel e jogo no lixo. Mas não tenho
nenhuma segurança pra amanhã.
\end{stanza}
\begin{stanza}
\qquad *
\end{stanza}
\begin{stanza}
\qquad arrogância.\verseline
\qquad moedas.
\end{stanza}
\end{poem}
\clearpage
\begin{poem}
\sequencetitle{11. População carcerária}
\poemsectiontitle{1.}
\begin{stanza}
mandam a gente estudar\verseline
mas o que a gente quer é só\verseline
fugir da fábrica ou da vassoura
\end{stanza}
\poemsectiontitle{2.}
\begin{stanza}
nossos presos não tiveram
a sorte a ousadia de um diploma
\quad alguns só se formam pela cela especial
\end{stanza}
\poemsectiontitle{3.}
\begin{stanza}
tempo pra pensar...
\end{stanza}
\end{poem}
\clearpage
\begin{poem}
\sequencetitle{12. Por fim}
\begin{stanza}
\qquad No princípio\verseline
\qquad \qquad era o nada.
\end{stanza}
\end{poem}

\par\null\vfill
\begin{flushleft}
16.8.15: Dissertação sobre o nada\\
\qquad O Nada.\\
é necessario escrever/vomitar. mas odeio vomitar, não da prazer. a cartomante errou o vaticínio. é necessário falar de Nada mas sem falar de nada. Oco. perfeitamente à vontade comigo. não. não se trata de vomitar. mas de conter o vômito. o automatismo: vamos voltar de novo a esse assunto? completamente cansado. buscando esses espaços em branco. antecipar algumas leituras da lista?
\end{flushleft}

\par\null\vfill
\begin{center}
\large{Não há um}\\
\Large{
LADO B}
\end{center}
\begin{flushright}
\large{
\textit{- “Atravessamos o presente de olhos vendados...”\\
- Milan Kundera}
}
\end{flushright}

\end{document}