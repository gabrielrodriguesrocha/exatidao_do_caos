\begin{poem}
\poemsubtitle{1.}
\begin{stanza}
encontra consigo\verseline
cego de luz do\verseline
espelho partido\verseline
de fim a fim
\end{stanza}
\begin{stanza}
essa coisa humana\verseline
mente se expondo\verseline
dois gumes na cara do cara
\end{stanza}
\begin{stanza}
que essa mente o dentro\verseline
e esconde o real\verseline
centro \quad nele
\end{stanza}
\poemsubtitle{1.1.}
\begin{stanza}
e por acaso corre\verseline
as mãos na pele\verseline
e acha\verseline
aquela cicatriz\verseline
um souvenir\verseline
invisível ate antes\verseline
e esquecido\verseline
que era assim doce\verseline
como a dor pode
\end{stanza}
\begin{stanza}
e o porquê-dor\verseline
volta em objeto\verseline
que não mente\verseline
a nostalgia\verseline
cortante
\end{stanza}
\begin{stanza}
ou mente\verseline
e corta ainda\verseline
mais \qquad mas\verseline
não aquele corte\verseline
de trépano não\verseline
é um corte que vai\verseline
aliviar o cérebro\verseline
da pressão
\end{stanza}
\begin{stanza}
é um \qquad de excesso\verseline
de pressão\verseline
negativa\verseline
que leva a tentar\verseline
uma saída
\end{stanza}
\begin{stanza}
assim como\verseline
sem alguma coerência\verseline
se passa do sujeito\verseline
para a consequência\verseline
nesse caso mordente
\end{stanza}
\begin{stanza}
de repente lembra\verseline
o que já tinha\verseline
pensado\verseline
pensa \quad quem sabe\verseline
uma bala na ideia\verseline
não implode\verseline
a cabeça?
\end{stanza}
\poemsubtitle{2.}
\begin{stanza}
entretanto\verseline
o quase é sempre\verseline
e tudo que existe
\end{stanza}
\begin{stanza}
e o quase é quase mais\verseline
vazio que o vazio
\end{stanza}
\begin{stanza}
exemplo\verseline
correr no mesmo\verseline
passo atras de quem\verseline
se move um passo\verseline
de cada vez
\end{stanza}
\begin{stanza}
exemplo\verseline
viver pensando e\verseline
dentro da cabeça\verseline
\qquad nunca fazer\verseline
\qquad que aconteça\verseline
\qquad o plano
\end{stanza}
\begin{stanza}
exemplo\verseline
viver sem risco e\verseline
com medo de chegar\verseline
a qualquer onde
\end{stanza}
\poemsubtitle{2.1.}
\begin{stanza}
estar a um passo\verseline
desse onde já é\verseline
estar a um\verseline
do poço\verseline
ou precipício
\end{stanza}
\begin{stanza}
morte que se morre\verseline
vivo
\end{stanza}
\begin{stanza}
sempre a um passo\verseline
é ate sem aonde\verseline
\qquad são outros quinhentos\verseline
\qquad quilômetros entre\verseline
\qquad você o alvo
\end{stanza}
\begin{stanza}
quem sempre a um passo\verseline
nunca esta mais perto\verseline
nem mais longe
\end{stanza}
\begin{stanza}
(...)
\end{stanza}
\end{poem}