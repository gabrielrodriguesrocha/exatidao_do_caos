\documentclass[20pt]{book}
\usepackage{fancyhdr,poemscol}
\usepackage[utf8]{inputenc}
\usepackage[margin=1.5in]{geometry}
\usepackage[brazilian]{babel}
\usepackage{marginnote}
\usepackage{calligra}
\usepackage{environ}
\usepackage{graphicx}
\usepackage{mathtools}
\usepackage{ragged2e}
\usepackage[normalem]{ulem}

\title{A EXATIDÃO DO CAOS}
\author{ANDRÉ MIRANDA SILVA}
\date{}
\global\verselinenumbersfalse

\begin{document}
\maketitle

\newpage
\thispagestyle{empty}
\mbox{}
\newpage

\vspace*{\fill}
\begin{center}\Large{ASTENIA}\end{center}
\begin{flushright}
\textit{
as.te.ni.a s.f. MED perda ou diminuição\\
da força física”\\
– Houaiss}
\end{flushright}
\vspace{\fill}

\clearpage
ASTENIA é essa eterna saída do caminho no meio dele. É esse
sempre parar com as coisas sem que nada tenha acontecido. É
encontrar muros e não subi-los. É ser impedido e unca impelido. É
sair do jogo sem começar a jogar. É a impotência, a impossibilidade
de encarar o desafio. É a perplexidade diante das coisas do mundo. A
incapacidade de apreender as informações e processá-las. É não
entender e não ser entendido. É sempre o equívoco e a insegurança
depois. Astenia é não viver.
\clearpage
\begin{poem}
\begin{stanza}
eu sei eu sei\verseline
que tem muito\verseline
que eu não sei\verseline
\qquad o que é melhor\verseline
\qquad olhar a lua cheia\verseline
\qquad ou fotografias\verseline
\qquad de abelhas?\verseline
o mundo bate\verseline
ou melhor\verseline
as coisas\verseline
\qquad eu não tô na reali\verseline
\qquad são muitos caminhos\verseline
\qquad e muitas escolhas\verseline
ou talvez\verseline
tudo é tão simples\verseline
que eu ainda não vi\verseline
\qquad não sei não sei\verseline
\qquad nem vivi
\end{stanza}
\begin{stanza}
\quad cansei.
\end{stanza}
\end{poem}
\clearpage
\begin{poem}
\begin{stanza}
mas enquanto descrevem a\verseline
estrutura do dia ou\verseline
a gramática da língua\verseline
em folhas de papel\verseline
\qquad folhas caem\verseline
\qquad das árvores\verseline
\qquad e não podem\verseline
\qquad ser faladas\verseline
as palavras são muito bonitas\verseline
mas \quad já não bastam
\end{stanza}
\end{poem}
\clearpage
\begin{poem}
\begin{stanza}
se ninguém viu \qquad ouviu\verseline
não existe\verseline
\qquad as folhas dos livros\verseline
\qquad as folhas das árvores\verseline
\qquad têm o mesmo valor\verseline
\qquad e são inúteis\verseline
se o sol explodisse\verseline
eu morreria\verseline
agora ou\verseline
em bilhões de anos\verseline
\qquad se eu morrer eu não tenho\verseline
\qquad \qquad nenhum plano b
 \end{stanza}
 \end{poem}
\clearpage
\begin{poem}
\begin{stanza}
preciso que me dê a mão\verseline
porque a vida é infinita\verseline
e o instante me assusta\verseline
\qquad ou todo mundo é ruim\verseline
\qquad ou sou só eu\verseline
que estou lá fora\verseline
ao meio dia\verseline
hora em que nada\verseline
nada circula\verseline
\qquad se eu dissesse\verseline
\qquad uma mentira\verseline
\qquad sei que não gostaria\verseline
a vida são só pastos\verseline
e pontos de ônibus\verseline
\qquad por isso vamos\verseline
\qquad correr correr\verseline
\qquad até cair\verseline
mesmo sem chegar\verseline
a nenhum lugar nenhum\verseline
\qquad o importante\verseline
\qquad é estar.
\end{stanza}
\end{poem}
\clearpage
\begin{poem}
\begin{stanza}
fala\verseline
é mais fácil\verseline
que a desfala\verseline
e mais difícil\verseline
que a não fala
\end{stanza}
\begin{stanza}
resvala no\verseline
querer dito\verseline
o que se\verseline
quer dizer
\end{stanza}
\begin{stanza}
e o que\verseline
se quer\verseline
se deve\verseline
nunca\verseline
interditar
\end{stanza}
\begin{stanza}
raiva\verseline
contra o mecanismo\verseline
um abissal\verseline
desejo\verseline
de um corpo\verseline
ou copo\verseline
d'água
\end{stanza}
\begin{stanza}
o abismo\verseline
da não fala\verseline
é vazio\verseline
e muitos são\verseline
os que o acham
\end{stanza}
\begin{stanza}
(sai dele\verseline
\ sai dele\verseline
\ meu povo)
\end{stanza}
\begin{stanza}
os que não\verseline
podem\verseline
o trabalho\verseline
de se abrir
\end{stanza}
\begin{stanza}
os que não\verseline
querem\verseline
o peso\verseline
de sorrir
\end{stanza}
\end{poem}
\clearpage
\begin{poem}
\begin{stanza}
As palavras são muito bonitas\verseline
mas já não bastam\verseline
\qquad a vida é vivida\verseline
\qquad em não viver\verseline
vidinha de rotinas\verseline
e conveniências\verseline
\qquad NINGUÉM MAIS GRITA\verseline
o poeta que vive em mim\verseline
dorme o dia inteiro\verseline
\qquad o deus que vive em mim\verseline
\qquad se desconverteu\verseline
nem sempre se pode ter\verseline
tudo que se quer\verseline
\qquad as cordas vibram\verseline
\qquad o baixo sobressai\verseline
\qquad a bateria explode\verseline
\qquad \qquad mas\verseline
ninguém grita mais
\end{stanza}
\end{poem}
\clearpage
\begin{poem}
\begin{stanza}
se me pegam na rua\verseline
já era eu\verseline
\qquad a boca se cala\verseline
\qquad e nisso acerta\verseline
a língua é muito falha\verseline
e não serve\verseline
pra defender\verseline
\qquad nem posso me esconder\verseline
\qquad por trás caneta
\end{stanza}
\end{poem}
\clearpage
\begin{poem}
\begin{stanza}
recebo\verseline
\qquad um olhar\verseline
\qquad um sorriso\verseline
e ando mais rápido\verseline
\qquad desvio
\end{stanza}
\begin{stanza}
medo\verseline
\qquad de realizar
\end{stanza}
\end{poem}
\clearpage
\begin{poem}
\begin{stanza}
se é pelo bem de todos\verseline
e a felicidade geral da nação\verseline
– mente sã\verseline
\quad corpo são –\verseline
\qquad passaremos sim\verseline
\qquad em frente a vocês\verseline
\qquad na primeira do plural\verseline
passaremos sim\verseline
de mãos dadas\verseline
de braços dados\verseline
de corpos grudados\verseline
\qquad passaremos sim\verseline
\qquad com fones de olvido\verseline
\qquad para esquecer com que frequência\verseline
\qquad vocês nos iludem\verseline
passaremos sim\verseline
de peitos abertos\verseline
e olhos lavados\verseline
para ver o fim\verseline
e o que vem\verseline
muito antes do começo\verseline
\qquad passaremos sim\verseline
\qquad todos juntos\verseline
\qquad para verem o quanto somos\verseline
\qquad diferentes e iguais\verseline
passaremos sim\verseline
ao vivo ao meio dia\verseline
para que vocês não precisem\verseline
nem ver seus jornais\verseline
\qquad passaremos sim\verseline
\qquad para que vejam\verseline
\qquad o quanto odiamos\verseline
\qquad uns aos outros\verseline
mas para que vejam\verseline
que entre nós\verseline
há sempre a certeza\verseline
do perdão
\end{stanza}
\end{poem}
\clearpage
\begin{poem}
\begin{stanza}
do alto das sarjetas\verseline
do fundo das calçadas\verseline
de dentro das carteiras\verseline
debaixo das janelas\verseline
\qquad silêncios\verseline
\qquad condições\verseline
\qquad cautelas\verseline
desde o passado até agora\verseline
reclamar não deu em nada\verseline
não é enredo de novela\verseline
os fracos não têm vez\verseline
se você fez algo\verseline
ninguém sabe que fez\verseline
é preciso mais que vida\verseline
mais que funções vitais\verseline
é preciso mais que arte\verseline
é preciso superar\verseline
\qquad vejo arte\verseline
\qquad em toda parte\verseline
\qquad vejo lixo\verseline
\qquad já vi morte\verseline
mas eu não vi\verseline
a solução\verseline
\qquad \qquad que é a força\verseline
\qquad \qquad \qquad que é a firme\verseline
\qquad \qquad resolução
\end{stanza}
\end{poem}
\clearpage
\begin{poem}
é apenas o começo\verseline
andamos calados\verseline
criando aqui dentro\verseline
uma desobediência\verseline
\qquad mas quem somos nós?\verseline
\qquad \qquad um pronome\verseline
\qquad \qquad rostos frágeis\verseline
a força que calamos\verseline
não vai explodir?\verseline
\hspace{5em}(que força?)\verseline
\qquad poderia ser\verseline
\qquad no papel\verseline
\qquad na tela\verseline
\qquad ou na voz\verseline
\hspace{5em} digital\verseline
mas as palavras\verseline
elas não bastam\verseline
\qquad não
\end{poem}
\clearpage
\hspace{4em}Poema artefato \textit{vs}. Poema discurso. Sem versus. Só versos. \\
\indent\hspace{4em}(Sem trocadilhos ruins como esse). Construir aquela rua de
que eu falei. De dentro pra fora. Montar a crocância da casca do pão. \\
\indent\hspace{4em}Emoção? Emoção. Construção. Como criar e ainda mostrar o
que está? Sem dilemas…(?) Sem tensões…(?) \\
\indent\hspace{12em}Poesia é problema (?)
\clearpage
\noindent Mas não posso entender que algo nasça do não. A fria negação. A luz
trêmula e pálida. Nada serve como alimento. As coisas têm de ser
vistas nas suas devidas proporções. Mesmo que assim seja: não ir à
festa. Rato de biblioteca. Fichas de cartolina. Mesmo que não seja
assim: diversão é solução, sim. É desse modo que se vive – através
dele. E seu muito poder. Sim, senhor. Através do sim.
\clearpage
\noindent As coisas escapam por entre os dedos. O mundo. O que acabou de
acontecer. Como andar de olhos fechados. O tempo é uma poeira
fininha. Inalcançável. Inatingível. Intangível. Todos os adjetivos.
Principalmente na rotina so-nâm-bu-la. A retina tão dificilmente
excitável. Olhar olhar e não ver nada. Viver sem saber. O mundo
foge. As coisas. Viver sem viver. Sem viver. Astenia.
\center{\textit{28.4.2015}}
\clearpage
\begin{poem}
\begin{stanza}
o sol refletidono concreto\verseline
machuca as rotinas\verseline
no passo sonâmbulo dos dias\verseline
as calçadas escorrem\verseline
debaixo dos pés que buscam\verseline
uma orfandade voluntária\verseline
aquela outra sozinhez\verseline
que é dentro de si\verseline
em meio a outros
\end{stanza}
\begin{stanza}
\qquad CONSERTA-SE:\verseline
\qquad celulares – tablets – PCs\verseline
e nada mais
\end{stanza}
\begin{stanza}
ninguém pode imaginar a paz\verseline
contida em um copo d'água
\end{stanza}
\end{poem}
\clearpage
\begin{poem}
\begin{stanza}
o ser transtornado\verseline
se devora em perguntas\verseline
surdas ao extremo:\verseline
o aqui dentro? o lá fora?\verseline
até onde chegar\verseline
à força de verdades?\verseline
e não é sua lembrança\verseline
que passa na janela\verseline
através de muitos metros\verseline
na neblina?\verseline
um segundo passageiro\verseline
se levanta\verseline
cansado de pensamentos repetidos\verseline
até o absurdo.\verseline
o tempo parado\verseline
absoluto\verseline
impede que as coisas\verseline
se dissolvam.\verseline
nem tudo se resolve\verseline
com falagens\verseline
(o general e sua\verseline
falange imperial\verseline
beberam o sangue\verseline
do inimigo –\verseline
beberam um pôr-de-sol\verseline
terra vermelha\verseline
vaso de argila)\verseline
as pontas soltas do passado\verseline
levantadas pelo vento\verseline
marcando os umbrais\verseline
das portas abertas do presente:\verseline
passaremos? ficaremos?\verseline
são dúvidas duplicadas\verseline
na lâmina dura da água\verseline
à beira da estrada
\end{stanza}
\end{poem}
\clearpage

\vspace*{\fill}
\begin{center}
\Large{
LONG PLAY\\
(365 rpm)}
\end{center}
\vspace{\fill}

\clearpage

\vspace*{\fill}
\begin{center}
\Large{
LADO A}
\end{center}
\begin{flushright}
\large{
\textit{- “Deem-me uma outra vida e estarei cantando...”\\
- Iósif Bródski}, Para minha filha
}
\end{flushright}
\vspace{\fill}

\begin{poem}
\sequencetitle{1. Intro}
\begin{stanza}
\qquad mal nasci\verseline
\qquad já planejo crimes\verseline
 – que eu traia\verseline
mas não seja nunca traído\verseline
por esta palavra:\verseline
\qquad (ou esta:\verseline
\qquad \qquad ou esta:\verseline
\qquad \qquad \qquad ou esta:)\verseline
 – que eu invada\verseline
mas não seja nunca\verseline
invadido por este pudor\verseline
\qquad \qquad este desejo escondido\verseline
\qquad \qquad de não viver\verseline
\qquad \qquad de sentar em cantos\verseline
\qquad \qquad \qquad de paredes\verseline
\qquad \qquad e responder o eco\verseline
\qquad \qquad \qquad da própria voz\verseline
\qquad – que eu tema\verseline
\qquad mas não seja nunca vítima\verseline
\qquad do medo dos outros\verseline
\qquad que tem calado a voz\verseline
\qquad dos nossos abraços\verseline
 – que eu roube\verseline
mas que nunca tirem de mim\verseline
o que eu tenho de eterno:\verseline
\quad as paredes do instante\verseline
\quad que bloqueiam\verseline
\quad que são maiores que os antes
\end{stanza}
\end{poem}
\clearpage
\begin{poem}
\sequencetitle{2. voyeur}
\poemsectiontitle{a.}
\begin{stanza}
suéter de losangos e óculos de coração\verseline
(alguma lolita com frio)\verseline
esperava o ônibus perto da rua\verseline
e sorriu \qquad juro por deus\verseline
de mostrar os dentes\verseline
por trás do batom vermelho
\end{stanza}
\begin{stanza}
me entristece perceber que eu descrevi a mulher\verseline
como quem descreve uma caixa de frutas\verseline
ou um copo com um resto de leite\verseline
em cima da mesa\verseline
me perco nessas voltas\verseline
mas nem aprendi\verseline
a usar as palavras)
\end{stanza}
\poemsectiontitle{b.}
\begin{stanza}
inutilmente esperei um milagre\verseline
de pé encostado na grade\verseline
enquanto o ônibus não vinha\verseline
ninguém se jogou no meu pescoço\verseline
ninguém ninguém nos meus braços
\end{stanza}
a ficção alimenta sonhos falsos\verseline
mas alimenta sonhos
\begin{stanza}
essas meninas têm o rosto impermeável\verseline
maquiagem a prova d'água e de teorias\verseline
antes da viagem \qquad antes de tudo\verseline
a poesia não tem a menor impotência\verseline
o poeta grita no livro fechado\verseline
mas além de livros um país\verseline
se faz de homens e mulheres\verseline
de mulheres e mulheres\verseline
de homens e homens\verseline
de palavras\verseline
de ideias\verseline
etc.
\end{stanza}
\poemsectiontitle{c.}
\begin{stanza}
dentro de si é uma mala 007\verseline
de que ninguém sabe o segredo\verseline
(exceto é claro aquele amigo\verseline
matemático mestre em combinatória\verseline
e convívio social)
\end{stanza}
\begin{stanza}
maleta dessas que se viola a tiro\verseline
mas eu não saio abrindo\verseline
os mistérios de ninguém\verseline
por muito menos já morri\verseline
por muito menos outros\verseline
já perderam o ponto
\end{stanza}
\begin{stanza}
uma pessoa que na vida\verseline
só chegou atrasada\verseline
por trocar o sim pelo não e vice-versa
\end{stanza}
\begin{stanza}
ando a pé
não corro o risco de ficar muito tempo\verseline
me prender a quem seja na calçada\verseline
e esquecer \qquad a verdade
\end{stanza}
\begin{stanza}
\center{a verdade \qquad a verdade \qquad a verdade}\verseline
\center{\textit{[[repeat]]}}
\end{stanza}
\end{poem}
\clearpage
\begin{poem}
\sequencetitle{3. objeto de desejo}
\begin{stanza}
enquanto você não está aqui\verseline
rugindo suas músicas de\verseline
pré-duplo-homicídio-suicídio\verseline
remastigando essas lembranças\verseline
remasterizadas\verseline
de um passado inútil\verseline
ou a nostalgia futura\verseline
de um tempo impossível\verseline
e descafeinado \qquad essas\verseline
fantasias que se vê em toda esquina
\end{stanza}
\begin{stanza}
enquanto você não está se lamentando\verseline
e eu não me lamento pra você\verseline
de não ter agarrado enquanto podia\verseline
todas as chances que o mundo\verseline
dava \quad dava \quad voltas e eu imóvel\verseline
bem como um móvel na sala\verseline
um sofá \qquad calado e útil\verseline
(você me chamava de\verseline
\qquad \qquad criado mudo\verseline
e eu não sabia o que era isso)
\end{stanza}
\begin{stanza}
enquanto você não está fazendo\verseline
seu habitual espetáculo\verseline
\qquad (a vida é um cinema em\verseline
\qquad \qquad dia de chuva)\verseline
ou torturando as pessoas com\verseline
sua voz de navalha na carne\verseline
ou fazendo ligações perigosas\verseline
\qquad uma vida-montanha-russa\verseline
ou contando vantagem e\verseline
histórias comoventes que mais parecem\verseline
piadas sem graça\verseline
ou servindo nossos olhares\verseline
de mais um exemplar\verseline
da sua antiarte inútil\verseline
ou dizendo futilidades\verseline
da sua prima ou daquela\verseline
sua amiga que bem que\verseline
podia ter morrido\verseline
ou do seu gato\verseline
que é só um pedaço gordo\verseline
de carne de cadáver
\end{stanza}
\begin{stanza}
enquanto você não vem\verseline
é como se eu fosse um estrangeiro\verseline
na minha própria vida
\end{stanza}
\end{poem}
\clearpage
\begin{poem}
\sequencetitle{4. definições}
\poemsectiontitle{1.}
\begin{stanza}
arte é o vazio refletido no espelho\verseline
enxergar através de lentes\verseline
antimiopemente \qquad fazer questão\verseline
de que a água seja bem peneirada\verseline
correr atrás do vento \qquad pra usar\verseline
uma referência clássica\verseline
é fazer com que o velho\verseline
pareça nascido agora\verseline
e reformar os olhos com catarata\verseline
reciclar os ouvidos dos surdos\verseline
deformar o que está aí\verseline
para que mudando tudo\verseline
se chegue à forma real das coisas
\end{stanza}
\poemsectiontitle{2.}
\begin{stanza}
ela disse\verseline
\qquad Tudo é arte\verseline
\qquad \qquad e eu ia começar a dizer\verseline
\qquad Não…\verseline
– ela me interrompeu com aquele olhar\verseline
que significa\verseline
\qquad Já vem você me chamar de burra
\end{stanza}
\begin{stanza}
\qquad (como se eu não fosse a carne\verseline
\qquad que diz sim pra tudo\verseline
\qquad aquele que é ofendido\verseline
\qquad e pede desculpas\verseline
\qquad o desprezível desprezado\verseline
\qquad que se humilha se rebaixa\verseline
\qquad para que os outros sejam\verseline
\qquad os glorificados\verseline
\qquad escondido debaixo\verseline
\qquad das solas dos príncipes\verseline
\qquad do mundo\verseline
\qquad esses outros que nunca\verseline
\qquad jamais levam porrada\verseline
\qquad os que apontam e riem\verseline
\qquad dos que só têm de seu\verseline
\qquad coisas emprestadas\verseline
\qquad – usadores de palavras)
\end{stanza}
\poemsectiontitle{3.}
\begin{stanza}
inútil dizer o que é o poema
\end{stanza}
\begin{stanza}
\qquad o poema é esse fazer e refazer o nada\verseline
\qquad \qquad \qquad \qquad do nada – 
\end{stanza}
\begin{stanza}
silêncios exaltados
\end{stanza}
\begin{stanza}
nem poucas nem mais palavras
\end{stanza}
\begin{stanza}
palavra.
\end{stanza}
\poemsectiontitle{4.}
\begin{stanza}
precisamos de algo mais que definições\verseline
precisamos de edificações de areia\verseline
de fortificações de ar\verseline
\qquad \qquad precisamos de sonhos antes de tudo\verseline
\qquad sonhos para realizar dar vender\verseline
ou enterrar no quintal de casa\verseline
\qquad os meus ideais estão num lugar bem seguro\verseline
enfiados onde ninguém vai pôr a mão\verseline
\qquad a minha segurança são os cadeados\verseline
e os cadeados dos cadeados\verseline
\qquad a minha segurança\verseline
é que hoje tudo é automático\verseline
(falo hoje como se houvesse o passado)\verseline
e todos podem sair sabendo que ao voltar\verseline
seus segredos estarão bem guardados\verseline
na boca dos amigos dos amigos dos amigos\verseline
dos conhecidos dos amigos dos conhecidos\verseline
dos ex e dos ex dos ex amigos\verseline
e nem digo nas bocas digo nos dedos\verseline
digo nas redes digo nos bytes dos sites\verseline
\qquad lugares onde a eternidade\verseline
\qquad é transitória
\end{stanza}
\end{poem}
\clearpage
\begin{poem}
\sequencetitle{5. Notas}
\poemsectiontitle{\textit{\underline{Ontem}}}
\begin{stanza}
Ouço os barulhos aí de fora e sofro. Ai.\verseline
Não adianta olhar pela janela que não vem ninguém.\verseline
Pensei que os diamantes fossem para sempre. Estava enganado.\verseline
Parece que eles mofam e apodrecem quando na sombra da\verseline
verdade jogada na cara.\verseline
Não adianta.\verseline
Não vem ninguém.
\end{stanza}
\poemsectiontitle{\textit{\underline{Há dez dias}}}
\begin{stanza}
A alegria do pão de milho contra as lâminas do álcool.\verseline
Sabor de cobre e fumaça.\verseline
Prevejo que vai começar tudo de novo.\verseline
Estamos preparados para a necessária fuga.\verseline
Imploro a Deus que seja mentira. Me ouviria?\verseline
Imploro que a verdade seja o sonho que eu tive ontem. Pai!\verseline
Como eu sofro!\verseline
Desenhei no chão com giz.\verseline
Um jogo. Um zigue-zague. Contra o tique-taque dos que me\verseline
compram e vendem. Absoluto. Frente a frente não sei falar. Só\verseline
abraços. Um absurdo.\verseline
Desse jeito que nos desespera. Dizes pera. Sinto a aflição de\verseline
seus olhos tão modernos. O que eles querem é o contrário do\verseline
que eu. Por isso sinto esta como que faca de açúcar quando\verseline
estou feliz contigo mas a felicidade não é completa porque por\verseline
mais que eu te toque e ouça você ainda fala uma língua outra.\verseline
Escrevo pra você sob uma rajada de silêncios, emoções\verseline
contrárias. Nunca lerá.\verseline
Mas eu insisto em ver flores e abelhas e lembrar.\verseline
Além disso o modo como você me faz sofrer e flutuar é \verseline
totalmente útil pra essa arte fútil.\verseline
E fatalmente não teremos nenhuma paz.\verseline
Nem rimas.\verseline
Querido diário.
\end{stanza}
\poemsectiontitle{\textit{\underline{Há vince e cinco dias}}}
\begin{stanza}
A última semana.\verseline
Sempre éramos idiotas antes de hoje.\verseline
Ou somos todos os dias mas o fato de ser hoje nos torna cegos\verseline
a essa idiotice.\verseline
O ano começa a acabar.\verseline
Tive que fazer essa tentativa. Se não der não deu e fazer o quê\verseline
\qquad seguir em frente ou em outra direção de modo a nem\verseline
sequer reste um vestígio dessa coisa absurda que se chama.
\end{stanza}
\begin{stanza}
\qquad Vale a pena ler o último volume?\verseline
\qquad Estou pensando em dar uma volta.\verseline
\qquad Ar.\verseline
\textit{(Ontem assisti a um filminho de adolescentes. Ilusões vãs.\verseline
Ricos e bonitos. Transgressão convencional. Vale nada)}.
\end{stanza}
\poemsectiontitle{\textit{\underline{Há trinta e um dias}}}
\begin{stanza}
Essa música me faz sentir insuportavelmente adolescente.\verseline
Insuportável \quad In-su-por-ta-vel-men-te.\verseline
O advérbio e-nor-me-men-te po-lis-sí-la-bo.\verseline
Nem de erva nem de solidão \qquad louco de som.\verseline
Menos lúcido que nunca.
\end{stanza}
\poemsectiontitle{\textit{\underline{Depois de amanhã}}}
\qquad All you need is love\verseline
\qquad \quad \  and all I need\verseline
\qquad \qquad \, \, is you\verseline
\qquad \qquad \quad \,    $<$3
\end{poem}
\clearpage
\begin{poem}
\sequencetitle{6. Notas 2}
\poemsectiontitle{a.}
\begin{stanza}
Essa insuficiência que eu sinto\verseline
essa proibição\verseline
direito negado\verseline
será algum\verseline
resto de passado?
\end{stanza}
\poemsectiontitle{b.}
\begin{stanza}
O silêncio é difícil\verseline
de apreender\verseline
As palavras para ele\verseline
são poucas
\end{stanza}
\poemsectiontitle{c.}
\begin{stanza}
Aprendo a viver\verseline
com lápis e borracha\verseline
e nunca mais\verseline
com a caneta definitiva
\end{stanza}
\end{poem}

\par\null
\qquad - \textit{fevereiro/março}
\clearpage
\begin{poem}
\sequencetitle{7. Encantado}
\begin{stanza}
\qquad pegar\verseline
\qquad \qquad um\verseline
\qquad \qquad \qquad atalho\verseline
\qquad para onde
\end{stanza}
\begin{stanza}
\qquad \qquad os sonhos\verseline
\qquad são poluções noturnas
\end{stanza}
\begin{stanza}
\qquad \qquad ou melhor\verseline
\textit{quando a manhã vem} com aquele\verseline
\qquad sorriso besta\verseline
\qquad \qquad você\verseline
\qquad \quad pega na mão dela\verseline
\qquad e vai\verseline
\qquad \qquad dar um passeio muito chato\verseline
\qquad \qquad \quad mas pode\verseline
\qquad porque sofrer é bom\verseline
\qquad \quad quando o sorriso\verseline
\qquad \qquad \, é bonito
\end{stanza}
\end{poem}
\clearpage
\begin{poem}
\sequencetitle{8. Maquinaria}
\begin{stanza}
planejar essas mentiras\verseline
com a perfeição\verseline
do possível
\end{stanza}
\end{poem}

\par\null
 - 21.4.2015
\clearpage
\begin{poem}
\sequencetitle{9. Choro}
\begin{stanza}
segunda-feira chuvosa e febril\verseline
\qquad e míope\verseline
\qquad e míope\verseline
\qquad e míope
\end{stanza}
\begin{stanza}
(fazer disso um drama)
\end{stanza}
\end{poem}
\clearpage
\begin{poem}
\sequencetitle{10. Notas 3}
\begin{stanza}
\qquad *
\end{stanza}
\begin{stanza}
Somos incapazes de perceber\verseline
que estamos aplaudindo\verseline
um ser abjeto?
\end{stanza}
\begin{stanza}
O que eu fiz em todo esse tempo não significa nada. Esse
tempo vazio. Intermezzo. Essa idade média da minha vida.
Essa nulidade. Amasso o papel e jogo no lixo. Mas não tenho
nenhuma segurança pra amanhã.
\end{stanza}
\begin{stanza}
\qquad *
\end{stanza}
\begin{stanza}
\qquad arrogância.\verseline
\qquad moedas.
\end{stanza}
\end{poem}
\clearpage
\begin{poem}
\sequencetitle{11. População carcerária}
\poemsectiontitle{1.}
\begin{stanza}
mandam a gente estudar\verseline
mas o que a gente quer é só\verseline
fugir da fábrica ou da vassoura
\end{stanza}
\poemsectiontitle{2.}
\begin{stanza}
nossos presos não tiveram
a sorte a ousadia de um diploma
\quad alguns só se formam pela cela especial
\end{stanza}
\poemsectiontitle{3.}
\begin{stanza}
tempo pra pensar...
\end{stanza}
\end{poem}
\clearpage
\begin{poem}
\sequencetitle{12. Por fim}
\begin{stanza}
\qquad No princípio\verseline
\qquad \qquad era o nada.
\end{stanza}
\end{poem}

\par\null\vfill
\begin{flushleft}
16.8.15: Dissertação sobre o nada\\
\qquad O Nada.\\
é necessario escrever/vomitar. mas odeio vomitar, não da prazer. a cartomante errou o vaticínio. é necessário falar de Nada mas sem falar de nada. Oco. perfeitamente à vontade comigo. não. não se trata de vomitar. mas de conter o vômito. o automatismo: vamos voltar de novo a esse assunto? completamente cansado. buscando esses espaços em branco. antecipar algumas leituras da lista?
\end{flushleft}

\vspace*{\fill}
\begin{center}
\large{Não há um}\\
\Large{
LADO B}\label{Lado B}
\end{center}
\begin{flushright}
\large{
\textit{- “Atravessamos o presente de olhos vendados...”\\
- Milan Kundera}
}
\end{flushright}
\vspace{\fill}

\center{
Epígrafes, Prefácio, Prelúdio ou\\
PRENÚNCIO PROFÉTICO DO QUE HÁ DE VIR\\}
\begin{poem}
\sequencetitle{DIANTE DA FOLHA BRANCA}
\begin{stanza}
Tanta lucidez da vertigem.\marginpar{\small{Van Gogh}}\verseline
Faz perder o pé na realidade.\verseline
Perder pé dentro de si mesmo,\verseline
sem contrapé, é uma voragem.
\end{stanza}
\begin{stanza}
Diante da folha branca e virgem,\verseline
na mesa, e de todo ofertada,\verseline
com medo de que ela sorvesse,\verseline
ei-lo, como louco, a estuprá-la.
\end{stanza}
\begin{stanza}
\hspace{2cm}*
\end{stanza}
\begin{stanza}
A folha branca é a tradução\marginpar{\small{Mallarmé}}\verseline
mais aproximada do nada.\verseline
Por que romper essa pureza\verseline
com palavra não nilpesada?
\end{stanza}
\begin{stanza}
A folha branca não aceita\verseline
senão a que acha que a merece:\verseline
essa so sobrevive ao fogo\verseline
desse branco que é gelo e febre.
\end{stanza}
\attribution{João Cabral de Melo Neto, in Agrestes (1981-1985)}
\end{poem}
\clearpage
\begin{poem}
\sequencetitle{DÚVIDAS APÓCRIFAS DE MARIANNE MOORE}
\begin{stanza}
Sempre evitei falar de mim,\verseline
falar-me. Quis falar de coisas.\verseline
Mas na seleção dessas coisas\verseline
não havera um falar de mim?
\end{stanza}
\begin{stanza}
Não havera nesse pudor\verseline
de falar-me uma confissão,\verseline
uma indireta confissao,\verseline
pelo avesso, e sempre impudor?
\end{stanza}
\begin{stanza}
A casa de que se falar\verseline
ate onde está pura ou impura?\verseline
Ou sempre se impõe, mesmo impura-\verseline
mente, a quem dela quer falar?
\end{stanza}
\begin{stanza}
Como saber, se há tanta coisa\verseline
de que falar ou não falar?\verseline
E se o evitá-la, o não falar,\verseline
é forma de falar da coisa?
\end{stanza}
\attribution{JCMN}
\end{poem}
\clearpage
\vspace*{\fill}
\begin{minipage}{\textwidth}
\center{“What is innocence, after all, if not the promise of future corruptibility?”}
\end{minipage}
\vspace{\fill}
\clearpage
\vspace*{\fill}
\begin{minipage}{\textwidth}
\center{
1\\
$*$ \\
I\\
---\\}

\[
\begin{array}{cccc}
8 & 16 & 24 & 32  \\
16 & 32 & 8 & 24 \\
24 & 8 & 16 & 32 \\
32 & 24 & 16 & 8\end{array}
%
 \begin{array}{c}
SATOR \\
AREPO \\
TENET \\
OPERA \\
ROTAS \\
\end{array}
\]
\end{minipage}
\vfill

\begin{flushleft}A perfeição vazia ou assimetria significativa? Um equilíbrio, mas nada clássico. Daí que partimos de uma inicial e, por meio de uma série de fatores, chegamos a uma desordem aparente (...)\end{flushleft}
\clearpage
\begin{poem}
\sequencetitle{A ESCULTURA DE MARY VIEIRA}
\begin{stanza}
dar a qualquer matéria\verseline
a aritmética de metal\verseline
dar lâmina ao metal\verseline
e à lâmina alumínio
\end{stanza}
\begin{stanza}
dar ao número ímpar\verseline
o acabamento do par\verseline
então ao número par\verseline
o assentamento do quatro
\end{stanza}
\begin{stanza}
dar a qualquer linha\verseline
projeto a pino de reta\verseline
dar ao círculo sua reta\verseline
sua racional de quadrado
\end{stanza}
\begin{stanza}
dar à escultura o limpo\verseline
de uma máquina de arte\verseline
por sua vez capaz da arte\verseline
de dar-se um espaço explícito
\end{stanza}
\attribution{JCMN}
\end{poem}
\clearpage
\vspace*{\fill}
\begin{minipage}{\textwidth}

\center{
$\star$ \\
\centering
\raggedright“Aqui se inicia\\
uma viagem clara\\
para a encantação”\\
\attribution{Ferreira Gullar}
$\star$
}
\end{minipage}
\vfill
\clearpage

\vspace*{\fill}
\begin{center}
\Large{
OS VERSOS PROIBIDOS}
\end{center}
\vspace{\fill}

\begin{poem}
\sequencetitle{1. 5h25min a.m.}
\begin{stanza}
um quarto quando acorda\verseline
é só vapor de suor\verseline
e água pesada de sonho\verseline
ruim \quad de pesadelo
\end{stanza}
\begin{stanza}
numa folha de papel\verseline
palavras que não dizem:\verseline
\qquad espelhos\verseline
\qquad só repetem\verseline
\qquad o que havia
\end{stanza}
\begin{stanza}
notas de suicídio forjadas\verseline
em madrugadas brancas\verseline
\qquad nunca vistas\verseline
ilustradas com umas\verseline
fotos falsas em p \& b\verseline
\qquad como a vida era\verseline
\qquad antigamente\verseline
monocromática monótona e chata
\end{stanza}
\begin{stanza}
o titulo “flores de ferro ---\verseline
pra que ninguém precise\verseline
engolir minha diarreia”
\end{stanza}
\begin{stanza}
e no criado mudo\verseline
sonhos registrados\verseline
toda vez que terminava\verseline
de fingir a própria morte
\end{stanza}
\end{poem}
\clearpage
\begin{poem}
\sequencetitle{2. Primeiro sonho}
\begin{stanza}
isso é uma profecia\verseline
prenúncio que veio\verseline
antes\verseline
prefário do que foi\verseline
e nunca vai voltar\verseline
prelúdio do que foi predito\verseline
e ainda faz eco\verseline
\qquad espalha\verseline
\qquad seus reflexos\verseline
\qquad pelas noites blues\verseline
\qquad \qquad --- e garante uns bons feels
\end{stanza}
\begin{stanza}
//
\end{stanza}
\begin{stanza}
sonhei que o mistério\verseline
se afastava de mim\verseline
e eu podia sentir\verseline
\qquad seu perfume\verseline
\qquad \qquad se despedaçando\verseline
e eu fechava minhas mãos\verseline
tentando prender o aroma\verseline
\qquad mas buscar manter\verseline
\qquad o que ia embora\verseline
\qquad por resolução firme e própria\verseline
\qquad \qquad era a pior cadeia\verseline
\qquad \qquad \quad a pior tortura
\end{stanza}
\begin{stanza}
//
\end{stanza}
\begin{stanza}
sonhei que o mistério\verseline
se abria\verseline
e se entragava a quem\verseline
pensasse o contrário\verseline
das ideias nele\verseline
\qquad \qquad a quem\verseline
não pudesse ser mais diferente\verseline
de si mesmo na frente do espelho\verseline
a quem menos soubesse ler\verseline
as entrelinhas\verseline
\qquad e a mim\verseline
\qquad \quad o único cego que sabe\verseline
\qquad \quad a cegeuria\verseline
\qquad \quad sua e dos que o cercam\verseline
\qquad o misterio virava a cara\verseline
\qquad e não queria toque\verseline
\qquad nem palavra
\end{stanza}
\begin{stanza}
//
\end{stanza}
\begin{stanza}
sonhei que o mistério se desfazia\verseline
assim que era tocado\verseline
e voltava a se juntar\verseline
quando eu dava um passo\verseline
\hspace{-1em} atrás\verseline
e eu fugia dessa luz\verseline
intermitente\verseline
para a segurança das coisas\verseline
concretas e acabadas\verseline
\hspace{-1em}as estradas e as praças\verseline
\hspace{-1em}retas e constantes
\end{stanza}
\end{poem}
\clearpage
\begin{poem}
\sequencetitle{3. Lá fora}
\begin{stanza}
andar na rua com livros\verseline
debaixo do braço\verseline
porque a leitura isso e aquilo\verseline
\qquad mas o mundo ainda escorre\verseline
\qquad por entre os dedos\verseline
\qquad \qquad pra que serve a mão\verseline
\qquad \qquad que não pode segurar\verseline
\qquad \qquad as coisas \quad suarazão?\verseline
os olhares e pernas cruzados\verseline
nos bancos em que\verseline
os cafés ao lado esfriam\verseline
\qquad esfriam as vontades\verseline
\qquad depois de alguns segundos\verseline
já é rotina andar no automático\verseline
aceitar a verdade absoluta\verseline
do outdoor da camiseta\verseline
\qquad conseguir um prazer\verseline
\qquad pouco ou nenhum\verseline
\qquad explorando a autotortura\verseline
o mendigo que cursou teologia\verseline
e pede um dinheiro pra cachaça\verseline
\qquad \qquad \qquad ganha\verseline
\qquad é um prêmio pela sinceridade\verseline
\qquad \qquad tem gente\verseline
\qquad \qquad que nunca diz o que quer\verseline
\qquad \qquad e não vale um cuspe\verseline
os carros correm\verseline
são o moto perpétuo\verseline
o motor que produz\verseline
o ar puro\verseline
com cheiro de cidade\verseline
\qquad a aparente desordem\verseline
\qquad dessa tarde\verseline
\qquad não é mais que um outro jeito\verseline
\qquad de arrumar as coisas\verseline
que eu morra\verseline
mas nunca veja demolida\verseline
aesquina onde eu sorri\verseline
e ganhei um sorriso\verseline
que esta queimando até agora
\end{stanza}
\end{poem}
\clearpage
\begin{poem}
\sequencetitle{4. Novamente dentro de si}
\begin{stanza}
vontade grande de traduzir\verseline
o de dentro no de fora\verseline
que anda e fala\verseline
\qquad mas é complexo o que separa\verseline
\qquad uma parte de outra parte\verseline
é tênue\verseline
que nem\verseline
uma gilete\verseline
e é espesso e alto como uma muralha\verseline
construída de verdades\verseline
\qquad das que importam\verseline
\qquad as que ninguém sabe.
\end{stanza}
\end{poem}
\clearpage

\vspace*{\fill}
\begin{minipage}{\textwidth}
\center{outro fim}\\
\center{x}\\
\center{outro sim}
\end{minipage}
\vspace{\fill}

\begingroup
\huge
\begin{poem}
\begin{stanza}
Letras soltas\verseline
\qquad \quad e sem\verseline
\qquad referente real\verseline
\qquad \quad sem ordem\verseline
\qquad o caso\verseline
\qquad \qquad sem régua\verseline
\qquad e o esquadro\verseline
\qquad \qquad ainda com a mão direita\verseline
(não significa nada)\verseline
\qquad não estudei semiótica\verseline
dispenso suas palavras\verseline
\qquad e seu ajutório\verseline
\qquad \qquad \qquad o caos\verseline
sem espaço para pensamento claro ordenado\verseline
que não seja digressão\verseline
sou obrigado a sempre mostrar essa face\verseline
serena \quad compenetrada mas é mentira\verseline
vou escrever direto e erro \quad coisa suja\verseline
palavra mijada que é pra isso que serve\verseline
ter caderno e caneta\verseline
\qquad \qquad \qquad se agora me dou à poesia\verseline
pensada \quad é aquela poesia que constroi\verseline
em torno de nada \quad toda a força do oco\verseline
que do vazio faz eco \quad mas com essa\verseline
poesia \quad que é o muito pensar em nada\verseline
e construir um labirinto \quad de paredes\verseline
rabiscadas no plano preterérito \quad medi-\verseline
das traçadas e calculadas com réguas\verseline
\qquad \qquad \qquad \qquad \qquad \qquad \qquad não\verseline
consigo escapar deste outro labirinto\verseline
e tenho que usar a palavra que é ideia\verseline
e não tijolo e escrever um como que\verseline
relato de como estou \quad algo que é só\verseline
jogado \quad quase aleatorio \quad labirinto tam-\verseline
bém \quad mas sem elegância e regra\verseline
\qquad caos também \qquad mas inexato\verseline
dança aleatoria \quad passos inventados na hora\verseline
hora \quad sem coreografia\verseline
\qquad \qquad \qquad \qquad \qquad \qquad \qquad e\verseline
se estou escrevendo aqui é porque pra isso\verseline
que serve um caderno: pra pensar e não\verseline
pra empilhar versos.
\end{stanza}
\end{poem}
\endgroup
\clearpage

\vspace*{\fill}
\begin{minipage}{\textwidth}
\center{\calligra{\Huge Caderno de Exercícios}}\\
\center{14 de julho de 2015}\\
\center{
Labirintos\\
Poemas para mim\\
$\overbrace{\text{(escritos)}}$
}\\
\center{\bfseries{NÃO-POESIA}}
\end{minipage}
\vspace{\fill}

\begin{poem}
\begin{stanza}
algo se quebrou\verseline
e não é nada\verseline
e por isso agora se começa\verseline
as coisas pelo meio\verseline
estou cansado de crer\verseline
nas mentiras que invento\verseline
não é como se eu existisse\verseline
ou devesse existir\verseline
já falei sobre isso\verseline
em outra ocasião\verseline
a impossibilidade de mim
\end{stanza}
\begin{stanza}
\rule{1cm}{0.4pt}\ \tiny{//}\ \rule{1cm}{0.4pt}
\end{stanza}
\begin{stanza}
é so não procurar\verseline
que ela virá\verseline
e assim eu fico na espera\verseline
de algo que não se sabe\verseline
e que eu não mereço\verseline
porque não me esforcei\verseline
é so não esperar\verseline
quando menos se espera\verseline
ela vem\verseline
e então eu fico querendo\verseline
sem a menor esperança\verseline
de que um dia venha mesmo\verseline
é so não querer\verseline
e assim eu não quero\verseline
o que mais desejo
\end{stanza}
\begin{stanza}
\rule{1cm}{0.4pt}\ \tiny{//}\ \rule{1cm}{0.4pt}
\end{stanza}
\begin{stanza}
Quantas vezes eu já disse\verseline
que é como se eu dormisse\verseline
e agora\verseline
estivesse acordado?
\end{stanza}
\begin{stanza}
E quantas vezes eu já disse\verseline
que é agora que eu acordo?
\end{stanza}
\begin{stanza}
Mas como pode alguém\verseline
que já estava acordado\verseline
acordar de novo tanto\verseline
tantas vezes?
\end{stanza}
\begin{stanza}
\rule{1cm}{0.4pt}\ \tiny{//}\ \rule{1cm}{0.4pt}
\end{stanza}
\begin{stanza}
Encher o papel do completo vazio\verseline
eco de nada\verseline
oco
\end{stanza}
\begin{stanza}
\rule{1cm}{0.4pt}\ \tiny{//}\ \rule{1cm}{0.4pt}
\end{stanza}
\begin{stanza}
Alguém como você me disse que não é bom\verseline
chorar.\verseline
Alguém como você\verseline
que tambem não sabe o que falar.\verseline
Alguém como você\verseline
que tambem não sabe o que fazer.
\end{stanza}
\begin{stanza}
Então eu estou andando so com os pés\verseline
e você ou alguém assim me disse\verseline
que não vale a pena andar so.
\end{stanza}
\begin{stanza}
Então eu estava cansado de falar de mim\verseline
e você ou outro alguém me deu\verseline
matéria nova para os meus poemas.
\end{stanza}
\begin{stanza}
E eu aprendi o valor dos diamantes\verseline
E eu aprendi o valor dos diamantes
\end{stanza}
\begin{stanza}
Alguém como você me ensinou\verseline
o valor dos diamantes
\end{stanza}
\begin{stanza}
Alguém como você\verseline
sem nem saber\verseline
me ensinou a viver.
\end{stanza}
\begin{stanza}
*
\end{stanza}
\end{poem}
\clearpage
\begin{poem}
\sequencetitle{Andrea Doria - Legião Urbana}
\begin{stanza}
Às vezes parecia\verseline
Que, de tanto acreditar\verseline
Em tudo que achávamos tão certo\verseline
Teríamos o mundo inteiro\verseline
E até um pouco mais\verseline
Faríamos floresta do deserto\verseline
E diamantes de pedaços de vidro
\end{stanza}
\begin{stanza}
Mas percebo agora que o teu sorriso\verseline
É indiferente, quase parecendo te ferir
\end{stanza}
\begin{stanza}
Não queria te ver assim\verseline
Quero a tua força como era antes\verseline
O que tens é só teu, e de nada vale fugir\verseline
E não sentir mais nada
\end{stanza}
\begin{stanza}
Às vezes parecia que era só improvisar\verseline
E o mundo então seria um livro aberto\verseline
Até chegar o dia em que tentamos ter demais\verseline
Vendendo fácil o que não tinha preço
\end{stanza}
\begin{stanza}
Eu sei, é tudo sem sentido\verseline
Quero ter alguém com quem conversar\verseline
Alguém que depois\verseline
Não use o que eu disse contra mim
\end{stanza}
\begin{stanza}
Nada mais vai me ferir\verseline
É que eu já me acostumei\verseline
Com a estrada errada que eu segui\verseline
E com a minha própria lei
\end{stanza}
\begin{stanza}
Tenho o que ficou\verseline
E tenho sorte até demais\verseline
Como eu sei que tens também.
\end{stanza}
\end{poem}
\clearpage
\begin{poem}
\begin{stanza}
Alguém como você\verseline
me disse que não é bom chorar\verseline
Alguém como você\verseline
que é apenas aprendiz\verseline
Alguém como você\verseline
que também não sabe o que fazer.
\end{stanza}
\begin{stanza}
Então eu estava andando fora do caminho\verseline
e você ou alguém assim me disse\verseline
que é bom não andar só\verseline
que é bom sair do vazio\verseline
e da escuridão.
\end{stanza}
\begin{stanza}
Eu estava cansado de tanto falar de mim\verseline
e você ou alguém assim\verseline
veio me animar\verseline
sentou do meu lado encostado no muro\verseline
com a mão na minha mão:\verseline
e ouviu cada silêncio.
\end{stanza}
\begin{stanza}
E eu aprendi o valor dos diamantes.
\end{stanza}
\begin{stanza}
Alguém, talvez você, me ensinou\verseline
que uma pedra bruta\verseline
ainda não mostrou seu valor.
\end{stanza}
\begin{stanza}
Você ou alguém igual\verseline
sem ter a intenção\verseline
me fez (vi)ver o sim\verseline
e esquecer o não.
\end{stanza}
\begin{stanza}
E eu aprendi o valor dos diamantes.
\end{stanza}
\end{poem}
\clearpage
\begin{poem}
\sequencetitle{“Tanta lucidez da vertigem”}
\begin{stanza}
Levanta ainda com vestígio\verseline
de absurdo fechando o olho\verseline
precisa uma bebida áspera
para afiar a lâmina\verseline
de ver com lucidez
\end{stanza}
\begin{stanza}
Anda em exercício\verseline
de enxergar as coisas coisas\verseline
não além não menos\verseline
Não se move de si senão para\verseline
branquear o branco\verseline
endurecer o diamante
\end{stanza}
\begin{stanza}
Fala o objeto em gesto reto\verseline
da mão\verseline
que varia o ângulo\verseline
e não faz arco\verseline
Quem sabe a agulha\verseline
de milhões de vértices\verseline
do redondo?\verseline
Do paradoxo umami\verseline
gosto ímpar\verseline
cinco\verseline
mas de aresta polida\verseline
sem excesso sem falta
\end{stanza}
\begin{stanza}
Para quem sai da cama\verseline
com um ritmo irritante\verseline
no ouvido\verseline
nada é mais tortura e desafogo\verseline
que encostar a orelha\verseline
em si mesmo e ouvir\verseline
os vários ecos do oco
\end{stanza}
\begin{stanza}
***
\end{stanza}
\end{poem}
\clearpage
\begingroup
\huge
\justify
O labirinto a ser criado é o labiritno da vertigem da linguagem. O labirinto da lucidez da palavra coisa. Com seus caminhos que se bifurcam. Por exemplo:$\left\{\begin{tabular}{c}uma coisa \\ outra coisa\end{tabular}\right .$, e nem sempre simétricos em espelho. E sempre a libertação do já feito e do alheio. Ainda que eu queira ser um eco ou continuador. É ainda em oposição que se faz essa continuação.
\begin{center}{\rule{4cm}{0.4pt}\raisebox{-0.5ex}{x}\rule{4cm}{0.4pt}}\end{center}
\begin{justify}
Oposição a tudo, sem matar nada nem ninguém. Mas tambem, trata-se aqui do so called sistema poético estabelecido de antemão, que é \underline{meu} sistema, não regras ditadas a ninguém. Regras so pra mim mesmo. O Tarik disse: estudar. Estudar até que se me posso talvez chamar poeta, ou escritor.
\end{justify}
\endgroup
\clearpage
\begin{poem}
\begin{stanza}
Os ricos gostam de dormir até tarde\verseline
apenas porque sabem que a corja\verseline
tem que dormir cedo para trabalhar de manhã\verseline
Essa é mais uma chance que eles\verseline
têm de ser diferentes:\verseline
parasitar,\verseline
desprezar os que suam para ganhar a comida,\verseline
dormir ate tarde,\verseline
tarde\verseline
um dia\verseline
ainda bem\verseline
demais.
\end{stanza}
\attribution{--- O Cobrador}
\end{poem}
\clearpage
\begin{poem}
\sequencetitle{Fractal}
\begin{stanza}
o todo tem todos\verseline
como tijolos
\end{stanza}
\begin{stanza}
coisa dentro da coisa\verseline
feita de ela mesma
\end{stanza}
\begin{stanza}
e mesmo os ocos\verseline
e os interstícios\verseline
são o algo\verseline
em negativo
\end{stanza}
\begin{stanza}
salientam\verseline
o recursivo
\end{stanza}
\begin{stanza}
para saber tudo\verseline
que tem que saber\verseline
\qquad \qquad tudo\verseline
antes de
\end{stanza}
\begin{stanza}
o infinito é\verseline
uma\verseline
cópia de si
\end{stanza}
\begin{stanza}
repetida repetida\verseline
sempre outra
\end{stanza}
\end{poem}
\clearpage
\begin{poem}
\poemsubtitle{1.}
\begin{stanza}
encontro com si mesmo\verseline
o espelho bipartido\verseline
de fim a fim
\end{stanza}
\begin{stanza}
humanamente um resto\verseline
uma cara de um cara\verseline
como se chama essa\verseline
espécie de objeto\verseline
sem utilidade prática
\end{stanza}
\begin{stanza}
e com as mãos\verseline
comentário da pele
\end{stanza}
\begin{stanza}
as nações guerreiras -\verseline
loucas de automatismo
\end{stanza}
\begin{stanza}
mas talvez uma pistola\verseline
e não tem volta\verseline
do pensamento
\end{stanza}
\poemsubtitle{1.}
\begin{stanza}
encontro consigo\verseline
espelho bipartido\verseline
cego de luz\verseline
de fim a fim
\end{stanza}
\begin{stanza}
essa coisa humana-\verseline
mente colocada:\verseline
dois gumes na carne\verseline
de cara \quad espécie\verseline
de objeto jamais\verseline
com utilidade pratica
\end{stanza}
\begin{stanza}
e por acaso corre\verseline
as mãos na pele\verseline
e acha\verseline
aquela cicatriz\verseline
em souvenir\verseline
de qualquer ontem?\verseline
invisível? talvez\verseline
tenha esquecido\verseline
que era assim doce
\end{stanza}
\begin{stanza}
assim como\verseline
sem alguma coerência\verseline
se passa do sujeito\verseline
para a consequência\verseline
neste caso mordente
\end{stanza}
\begin{stanza}
de repende blue devils\verseline
pensa \quad e quem sabe\verseline
uma pistola\verseline
implode a cabeça.
\end{stanza}
\poemsubtitle{2.}
\begin{stanza}
entretanto\verseline
um quase é sempre\verseline
e tudo que existe
\end{stanza}
\begin{stanza}
e o quase é mais\verseline
vazio que o vazio
\end{stanza}
\begin{stanza}
por isso o termo\verseline
explodir pra dentro\verseline
que so ocorre com\verseline
o que é cheio de oco
\end{stanza}
\poemsubtitle{3.}
\begin{stanza}
estar a um passo\verseline
é sempre do poço\verseline
ou precipício
\end{stanza}
\begin{stanza}
milhares de\verseline
quilômetros\verseline
entre si e si
\end{stanza}
\end{poem}
\begin{center}{\rule{4cm}{0.4pt}\raisebox{-0.5ex}{x}\rule{4cm}{0.4pt}}\end{center}
\clearpage
\begin{poem}
\poemsubtitle{1.}
\begin{stanza}
encontro consigo\verseline
cego luz do\verseline
espelho bipartido\verseline
de fim a fim
\end{stanza}
\begin{stanza}
essa coisa humana\verseline
mente se expondo\verseline
em dois gumes na cara\verseline
da cara \quad espécie\verseline
de objetos jamais\verseline
com utilidade
\end{stanza}
\begin{stanza}
e por acaso corre\verseline
as mãos na pele\verseline
e acha\verseline
aquela cicatriz\verseline
um souvenir?\verseline
esquecido\verseline
que era assim doce
\end{stanza}
\begin{stanza}
assim como\verseline
sem alguma coerência\verseline
se passa do sujeito\verseline
para a consequência\verseline
mordente nesse caso
\end{stanza}
\begin{stanza}
de repente blue devils\verseline
pensa \quad e quem sabe\verseline
uma bala na ideia\verseline
não impediria\verseline
a cabeça
\end{stanza}
\poemsubtitle{2.}
\begin{stanza}
entretanto\verseline
o quase é sempre\verseline
e tudo que existe
\end{stanza}
\begin{stanza}
o quase é mais\verseline
vazio que o vazio
\end{stanza}
\begin{stanza}
exemplo
\end{stanza}
\begin{stanza}
correr no mesmo\verseline
passo\verseline
atrás de quem\verseline
se move um passo\verseline
de cada vez
\end{stanza}
\begin{stanza}
exemplo
\end{stanza}
\begin{stanza}
estar a um paço\verseline
só de poço\verseline
ou precipício
\end{stanza}
\begin{stanza}
exemplo
\end{stanza}
\begin{stanza}
\underline{sempre} a um passo\verseline
são outros quinhentos\verseline
quilômetros entre\verseline
quem e quem
\end{stanza}
\end{poem}
\clearpage
\begin{poem}
\poemsubtitle{1.}
\begin{stanza}
encontra consigo\verseline
cego de luz do\verseline
espelho partido\verseline
de fim a fim
\end{stanza}
\begin{stanza}
essa coisa humana\verseline
mente se expondo\verseline
dois gumes na cara do cara
\end{stanza}
\begin{stanza}
que essa mente o dentro\verseline
e esconde o real\verseline
centro \quad nele
\end{stanza}
\poemsubtitle{1.1.}
\begin{stanza}
e por acaso corre\verseline
as mãos na pele\verseline
e acha\verseline
aquela cicatriz\verseline
um souvenir\verseline
invisível ate antes\verseline
e esquecido\verseline
que era assim doce\verseline
como a dor pode
\end{stanza}
\begin{stanza}
e o porquê-dor\verseline
volta em objeto\verseline
que não mente\verseline
a nostalgia\verseline
cortante
\end{stanza}
\begin{stanza}
ou mente\verseline
e corta ainda\verseline
mais \qquad mas\verseline
não aquele corte\verseline
de trépano não\verseline
é um corte que vai\verseline
aliviar o cérebro\verseline
da pressão
\end{stanza}
\begin{stanza}
é um \qquad de excesso\verseline
de pressão\verseline
negativa\verseline
que leva a tentar\verseline
uma saída
\end{stanza}
\begin{stanza}
assim como\verseline
sem alguma coerência\verseline
se passa do sujeito\verseline
para a consequência\verseline
nesse caso mordente
\end{stanza}
\begin{stanza}
de repente lembra\verseline
o que já tinha\verseline
pensado\verseline
pensa \quad quem sabe\verseline
uma bala na ideia\verseline
não implode\verseline
a cabeça?
\end{stanza}
\poemsubtitle{2.}
\begin{stanza}
entretanto\verseline
o quase é sempre\verseline
e tudo que existe
\end{stanza}
\begin{stanza}
e o quase é quase mais\verseline
vazio que o vazio
\end{stanza}
\begin{stanza}
exemplo\verseline
correr no mesmo\verseline
passo atras de quem\verseline
se move um passo\verseline
de cada vez
\end{stanza}
\begin{stanza}
exemplo\verseline
viver pensando e\verseline
dentro da cabeça\verseline
\qquad nunca fazer\verseline
\qquad que aconteça\verseline
\qquad o plano
\end{stanza}
\begin{stanza}
exemplo\verseline
viver sem risco e\verseline
com medo de chegar\verseline
a qualquer onde
\end{stanza}
\poemsubtitle{2.1.}
\begin{stanza}
estar a um passo\verseline
desse onde já é\verseline
estar a um\verseline
do poço\verseline
ou precipício
\end{stanza}
\begin{stanza}
morte que se morre\verseline
vivo
\end{stanza}
\begin{stanza}
sempre a um passo\verseline
é ate sem aonde\verseline
\qquad são outros quinhentos\verseline
\qquad quilômetros entre\verseline
\qquad você o alvo
\end{stanza}
\begin{stanza}
quem sempre a um passo\verseline
nunca esta mais perto\verseline
nem mais longe
\end{stanza}
\begin{stanza}
(...)
\end{stanza}
\end{poem}
\clearpage
\begin{poem}
\sequencetitle{A SAUCERFUL OF SECRETS}
\begin{stanza}
o dia todo ouvindo\verseline
música psicodelica
\end{stanza}
\begin{stanza}
calar a mente
\end{stanza}
\begin{stanza}
entrar num labirinto\verseline
pra fugir de outro\verseline
de viver\verseline
\quad fazer coisas
\end{stanza}
\begin{stanza}
e aqui dentro\verseline
um raciocínio muito denso\verseline
mas na rua\verseline
não vale um cuspe
\end{stanza}
\begin{stanza}
quatro horas escrevendo\verseline
trinta e nove palavras\verseline
sempre as mesmas
\end{stanza}
\begin{stanza}
e todas apenas eco\verseline
de digressões\verseline
sobre o nada
\end{stanza}
\begin{stanza}
por que pensar tanto\verseline
e tanto nesse nada?
\end{stanza}
\begin{stanza}
mas novamente\verseline
uma boa teoria\verseline
das coisas
\end{stanza}
\begin{stanza}
que mais uma vez\verseline
não serve nem\verseline
pra impressionar ninguém
\end{stanza}
\begin{stanza}
tortura mesmo\verseline
é ter esses olhos lavados
\end{stanza}
\begin{stanza}
\raisebox{-1.19ex}{*} \rule{1cm}{0.4pt} \raisebox{-1.19ex}{*} \rule{1cm}{0.4pt} \raisebox{-1.19ex}{*} \rule{1cm}{0.4pt} \raisebox{-1.19ex}{*} \rule{1cm}{0.4pt} \raisebox{-1.19ex}{*} \rule{1cm}{0.4pt} \raisebox{-1.19ex}{*}
\end{stanza}
\end{poem}
\vfill
\begin{justify}Mas se a proposta era falar de coisas, por que falar de nada? Ou chegar ao nada com coisas? Esse é o verdadeiro labirinto?\end{justify}
$\underbrace{\qquad \qquad \qquad}$
\clearpage
\begin{poem}
\begin{stanza}
onde já se viu um herói\verseline
andar na rua com\verseline
fones de ouvido?\verseline
com as mãos nos bolsos\verseline
cabeça baixa\verseline
fora da faixa\verseline
correndo riscos.
\end{stanza}
\begin{stanza}
onde já se viu um herói\verseline
tao sem heroísmo?\verseline
sem objetivo nem sentido na vida.
\end{stanza}
\begin{stanza}
só umas perguntas.\verseline
prevendo o futuro (porque eu posso):\verseline
na segunda vai ganhar\verseline
um quarto de doce.\verseline
já conheço coisa mais quente.
\end{stanza}
\begin{stanza}
felicidade é uma palavra\verseline
mas nunca pode dizer isso\verseline
que não aceitam teorias.
\end{stanza}
\begin{stanza}
um lema tatuado na mente:\verseline
selflessness and\verseline
no martyrization\verseline
but we all know that\verseline
selflessness is so selfish...
\end{stanza}
\end{poem}
\clearpage
\begin{poem}
\sequencetitle{XVII}
\begin{stanza}
Todos irão sempre contra ti\verseline
porque tens pureza.
\end{stanza}
\begin{stanza}
Porque o agitado de tuas mãos\verseline
é quase nostálgico.
\end{stanza}
\begin{stanza}
Porque tens olhos\verseline
ficarao abertos\verseline
para quem os vius\verseline
uma única vez.
\end{stanza}
\begin{stanza}
Todos irão sempre contra ti\verseline
porque hás de querer\verseline
um mundo novo e diferente.\verseline
Porque és estranho\verseline
e diferente para o nosso mundo.
\end{stanza}
\begin{stanza}
És quase um louco\verseline
porque não dás atenção\verseline
à toda gente.
\end{stanza}
\begin{stanza}
Dirão que és poeta.\verseline
Porque a poesia aparece nos teus gestos\verseline
como aparece fe na oração de um crente.\verseline
Mas o amor agora é tão difícil.
\end{stanza}
\begin{stanza}
Não existes para mim.\verseline
Mas agitado, febril,\verseline
quase doente, és vivo...
\end{stanza}
\begin{stanza}
Vivo demais para viver conosco.
\end{stanza}
\end{poem}
\clearpage
\begin{poem}
\begin{stanza}
Eu não quero outra coisa\verseline
só quero uma
\end{stanza}
\begin{stanza}
Que isso que queima fosse verdade!\verseline
Se fosse. Não sei\verseline
o que seria
\end{stanza}
\poemsubtitle{b.}
\begin{stanza}
só uma coisa \qquad a maior delas\verseline
a coisaagora \qquad verdadiamante\verseline
ou melhor \qquad se chama\verseline
chama \qquad coisa que queima\verseline
há duas \qquad uma\verseline
forte e antiga \qquad outra\verseline
branda \qquad nem se desenvolveu\verseline
\qquad toda boa lâmina\verseline
é lisa \qquad e doce\verseline
isso não \qquad é so e apenas\verseline
acidente\verseline
\qquad \qquad tropeço
\end{stanza}
\poemsubtitle{c.}
\begin{stanza}
a coisa que queima se chama\verseline
\qquad \qquad \qquad \qquad \qquad chama\verseline
coisa que está no centro\verseline
\qquad \qquad \qquad \qquad \qquad dentro das entranhas\verseline
inflamada entramada nas fibras\verseline
entranhada e estranha\verseline
\qquad há essa coisa sem palavra\verseline
\qquad e sem chamar \qquad so chamada\verseline
\qquad por um algo não nome\verseline
\qquad desejo mortalmente\verseline
o nome da agora coisa é também absurdo\verseline
como pode o fogo?\verseline
\qquad \qquad \qquad \qquad \qquad o que queima até consumir\verseline
ora se isso destrói \qquad como desejar com\verseline
chama? \qquad como querer até a morte?\verseline
se depois não se tem?
\end{stanza}
\poemsubtitle{d.}
\begin{stanza}
a agora coisa que levo se chama\verseline
chama\verseline
coisa entramada na entranha\verseline
entranhada nas fibras\verseline
tem uma estranha palavra não nome\verseline
desejar até a morte
\end{stanza}
\poemsubtitle{e.}
\begin{stanza}
a coisa de que falo é a coisa\verseline
que faz eco no escuro\verseline
a coisa de que falo é\verseline
o instrumento da vitoria\verseline
\qquad nome de mulher\verseline
isto conseguido sem esforço\verseline
isto de que falo\verseline
a fala
\end{stanza}
\poemsubtitle{f.}
\begin{stanza}
a coisa de que falo\verseline
é um vômito contido\verseline
essa coisa que queima se chama\verseline
chama da entranha\verseline
o estranho não nome\verseline
do desejar calado
\end{stanza}
\rotatebox[origin=c]{180}
{
\begin{minipage}[t]{\textwidth}
\begin{stanza}
essa coisa de que falo\verseline
é também o que desejo\verseline
e calo\verseline
uso uma palavra\verseline
não nome\verseline
para referir o que está dentro\verseline
e queima\verseline
e é só uma pena\verseline
um peso
\end{stanza}
\end{minipage}
}\\
\poemsubtitle{g.}
\begin{stanza}
aquilo de que falo\verseline
é o que tanto sabem\verseline
não preciso dizer um nome\verseline
para mostrar o que melhor\verseline
se compreende no escuro\verseline
falo do que esta do outro lado\verseline
e é inalcançável
\end{stanza}
\rotatebox[origin=c]{180}
{
\begin{minipage}[t]{\textwidth}
\begin{stanza}
ou é alcançável por alguns não mim\verseline
\qquad ADENDUM\verseline
\qquad APPENDIX\verseline 
a palavra escrita no espelho\verseline
as palavras que não fazem\verseline
sentido juntas \qquad o que não trabalho\verseline
foi outro que disse \qquad eu queria ter es-\verseline
crito um livro \qquad DO DESEJO\verseline
do deserto do não beijo essa palavra
\end{stanza}
\end{minipage}
}\\
\poemsubtitle{h.}
\begin{stanza}
essa palavra que eu não quero dizer\verseline
\qquad DESEJO\verseline
essa palavra que queima em mim e se chama\verseline
chama \qquad ainda que estivesse no deserto\verseline
mais seco \qquad minha boca estaria úmida\verseline
de querer \qquad ainda que estivesse exilado\verseline
na sibéria \qquad meu corpo estaria quente\verseline
de querer com fogo \quad fervor de sangue ardente\verseline
ou melhor querer com fervor de crente\verseline
e uma febre de doente
\end{stanza}
\poemsubtitle{h.0.}
\begin{stanza}
essa palavra que não quero dizer\verseline
\qquad DESEJO\verseline
isso que queima e se chama\verseline
chama \qquad reescrevi todos os livros\verseline
que tinha lido\verseline
para que alguém me saiba\verseline
ponho meu nome no que não é meu\verseline
nem eu mesmo\verseline
me reconheço \qquad só sei\verseline
de uma coisa entranhada\verseline
nas fibras de mim\verseline
usei palavras graves\verseline
gravadas a ferro\verseline
mas so quero saber de entrar\verseline
cada vez mais no labirinto\verseline
eu quero eu quero eu quero\verseline
nem eu mesmo\verseline
me enxergo no que sou\verseline
olho nas caras dos outros\verseline
tentando me encontrar\verseline
mas tudo isso no princípio apenas\verseline
queria dizer \qquad olhos azuis\verseline
depois \qquad tênis vermelhos\verseline
depois \qquad nada \quad que é o que\verseline estava
e ainda está\verseline
e não se escapa\verseline
nunca.
\end{stanza}
\poemsubtitle{h.1.}
\begin{stanza}
mas essa palavra que não quero dizer\verseline
\qquad DESEJO\verseline
isso que queima e se chama\verseline
chama\verseline
são tênis vermelhos em sua dança\verseline
e calças tambem\verseline
e não é ninguém\verseline
\qquad é a dança vermelha\verseline
\qquad dos tênis nos pés\verseline
\qquad \qquad calcanhares sao nomes\verseline
\qquad \qquad que giram o corpo\verseline
\qquad \qquad \qquad eu usaria até o termo\verseline
\qquad \qquad \qquad gracioso\verseline
\qquad \qquad \qquad \qquad ou outro um\verseline
\qquad \qquad \qquad \qquad que não encontro\verseline
isso quero e quero e o que eu quero\verseline
se chama\verseline
fogo inflamado \qquad \qquad dentro de mim\verseline
por essa dança
\end{stanza}
\poemsubtitle{h.2.}
\begin{stanza}
ou por outra dança\verseline
que são os olhos \qquad o uso próprio\verseline
de dois pra la dois pra cá \quad a expressão\verseline
perfeitamente burra\verseline
o obstáculo à leitura do que seja\verseline
em olhos \qquad é o desviar\verseline
sem nunca encontrar de novo
\end{stanza}
\poemsubtitle{h.3.}
\begin{stanza}
mas isso que quero\verseline
e temo\verseline
esse bale da cor\verseline
de notas baixas\verseline
isso não me infala\verseline
e não \quad não se chama\verseline
chama \quad pelo contrario\verseline
se apela por outra palavra\verseline
que é uma palavra\verseline
nunca usada\verseline
o mesmo nome\verseline
do homem
\end{stanza}
\poemsubtitle{h.4.}
\begin{stanza}
havia também um anjo\verseline
mas só cri por um dia\verseline
voltei ao ateísmo de sempre\verseline
e ao cinismo\verseline
quando de repente\verseline
de repente
\end{stanza}
\poemsubtitle{h.5.}
\begin{stanza}
mas isso que quero\verseline
e temo ter\verseline
essa dança da cor\verseline
de sangue de artéria\verseline
não tem nome de fogo\verseline
nem de gente\verseline
nem de qualquer\verseline
matéria conhecida\verseline
é um algo que não\verseline
o que me queima\verseline
e não se conhece por\verseline
labareda\verseline
é o impossivel\verseline
o zero ao invés\verseline
o tudo
\end{stanza}
\poemsubtitle{h.6.}
\begin{stanza}
mas agora que finalmente falei\verseline
de danças de olhos de tênis\verseline
e tais coisas\verseline
é como se
\end{stanza}
\poemsubtitle{h.6.1.}
\begin{stanza}
mas isso não é.\verseline
simplesmente não.
\end{stanza}
\poemsubtitle{i.}
\begin{stanza}
essa palavra que não quero dizer\verseline
\qquad \sout{DESEJO}\verseline
tem o mesmo nome do medo\verseline
o que é intenso\verseline
um vazio no centro
\end{stanza}
\poemsubtitle{j.}
\begin{stanza}
saber nomes de poetas\verseline
e recitar versos\verseline
de madrugada\verseline
em voz alta\verseline
eu sentia que estava\verseline
cheio de vida\verseline
e por isso\verseline
estava morrendo
\end{stanza}
\poemsubtitle{k.}
\begin{stanza}
caótico disperso\verseline
desorganizado \qquad assim\verseline
o que corre na chuva\verseline
para se sentir vivo\verseline
se estar vivo\verseline
é estar molhado \qquad assim\verseline
com a mente em parafuso\verseline
é uma confusa\verseline
epifaina negativa \qquad déjà vu\verseline
de um déjà vu \qquad quem\verseline
já sonhou\verseline
um sonho dentro de um\verseline
\qquad sabe do que eu falo\verseline
o pavor do inescapável\verseline
\qquad quem conhece a\verseline
\qquad paralisia do sono\verseline
quem lutou com deus\verseline
que o não deixava\verseline
dormir nem acordar\verseline
\qquad quem pensa em\verseline
\qquad arte\verseline
(esta uma é menos cabral)\verseline
essas coisas que saem num vômito\verseline
eu queria não escrever\verseline
como quem mija\verseline
\qquad \qquad \qquad antes\verseline
\qquad \qquad \qquad \qquad eu queria\verseline
viver como quem mija\verseline
não como quem\verseline
pede desculpa
\end{stanza}
\poemsubtitle{l.}
\begin{stanza}
o que eu farei\verseline
quando acabarem as letras?\verseline
al-kawarizmi
\end{stanza}
\poemsubtitle{m.}
\begin{stanza}
mas falamos daquilo\verseline
\qquad \qquad secreto\verseline
mesmo quando revelado\verseline
desejar o objeto\verseline
\qquad de desejo\verseline
e ser dele o\verseline
\qquad de desprezo\verseline
ainda que não o despreze\verseline
e até o queira bem\verseline
o problema aqui é\verseline
a proporção\verseline
uma questão\verseline
de não tanto quanto
\end{stanza}
\poemsubtitle{n.}
\begin{stanza}
ecos de fracassos. \qquad caminhando\verseline
sobre destroços de pretensão\verseline
\qquad o orgulho por um segundo\verseline
\qquad logo em cacos\verseline
vestígios de ter tentado algo\verseline
tentado errado\verseline
\qquad \qquad *\verseline
amar a humilhação e a tortura\verseline
e não tirar nada de bom\verseline
dessas duas \qquad so sentir\verseline
a crueldade\verseline
de quem se esforça em rir\verseline
cotidianamente o desrespeito\verseline
ao que de precioso\verseline
se oferece\verseline
\qquad \qquad *
\end{stanza}
\begin{stanza}
não era disso que se falava
\end{stanza}
\begin{stanza}
é uma pena ter que falar em tempo
\end{stanza}
\begin{stanza}
estar no tempo
\end{stanza}
\begin{stanza}
quando ele não é nada\verseline
mais que uma água
\end{stanza}
\begin{stanza}
um desespero
\end{stanza}
\begin{stanza}
livro escrito em estilo muito pobre\verseline
o anteriormente anunciado
\end{stanza}
\begin{stanza}
são só palavras \qquad depois \qquad antes
\end{stanza}
\begin{stanza}
\qquad \qquad *
\end{stanza}
\begin{stanza}
lembro ainda quando a vida\verseline
era inteira\verseline
feita de merda
\end{stanza}
\poemsubtitle{o.}
\begin{stanza}
mas essa palavra que não quero dizer\verseline
isso que queima e se chama\verseline
chama\verseline
são frases e poemas\verseline
em sua dança\verseline
e a memória de que\verseline
ainda há vida\verseline
mesmo que a morte\verseline
esteja caótia \quad em parafuso\verseline
mesmo sem certeza nenhuma\verseline
é so insegurança\verseline
e medo de ficar\verseline
sozinho para sempre\verseline
e sem ninguém\verseline
e nunca provar\verseline
o que há de bom\verseline
\qquad \qquad \qquad mesmo que \qquad nada\verseline
a grande mentira\verseline
não há possível\verseline
justificativa
\end{stanza}
\poemsubtitle{p.}
\begin{stanza}
não vejo nada\verseline
não vejo a fita\verseline
dominada\verseline
\qquad eu vejo os preto\verseline
\qquad sempre triste\verseline
\qquad nos canto do mundão
\end{stanza}
\attribution{--- Mano Brown}
\poemsubtitle{q.}
\begin{stanza}
que horrível é poder não usar as palavras\verseline
ainda que elas existam!
\end{stanza}
\poemsubtitle{u.}
\begin{stanza}
folha seca num\verseline
vendaval\verseline
um inútil:\verseline
é morrer aos poucos\verseline
eu me sentia assim,\verseline
tio
\end{stanza}
\attribution{--- Mano Brown}
\poemsubtitle{v.}
\begin{stanza}
escrevendo\verseline
a sensação constante\verseline
de estar pondo merda\verseline
no papel
\end{stanza}
\poemsubtitle{w.}
\begin{stanza}
lembra do espírito\verseline
do 14 de julho?\verseline
ele é tão vazio\verseline
quanto ser enrolado\verseline
pelas fantasias
\end{stanza}
\begin{stanza}
as madrugadas\verseline
dedicadas\verseline
ao nada
\end{stanza}
\begin{stanza}
aquilo que se chama\verseline
loucura\verseline
de querer o que não tem
\end{stanza}
\begin{stanza}
e o que\verseline
sabendo disso\verseline
se mantém perto\verseline
sem estar totalmente
\end{stanza}
\begin{stanza}
creio que aqui voltamos\verseline
ao espírito \verseline
do primeiro ensaio\verseline
deste livro
\end{stanza}
\begin{stanza}
o que é menor arte\verseline
que isto?
\end{stanza}
\begin{stanza}
ter dezesseis anos\verseline
sentir-se ridículo\verseline
algo supera?
\end{stanza}
\poemsubtitle{x.}
\begin{stanza}
acabei de inventar\verseline
uma nova fantasia:\verseline
voz aguda\verseline
meu deus\verseline
tenho que trabalhar\verseline
nada me é dado\verseline
e nem permitem\verseline
o suicídio:\verseline
\qquad quando mais eu escrevo\verseline
\qquad menos estou escrevendo\verseline
de natural já não digo nada\verseline
agora nem mais construo\verseline
\quad só estou lembrando\verseline
\qquad \qquad \qquad \qquad O QUÊ?\verseline
\rule{4cm}{0.4pt} como é\verseline
possivel ser poeta\verseline
sendo tão egoísta? \underline{ } \underline{ } \underline{ } \underline{ } \underline{ } \underline{ } \underline{ }
\end{stanza}
\clearpage
\poemsubtitle{y.}
\begin{stanza}
One of these days\verseline
I’m going to cut you\verseline
into little pieces
\end{stanza}
\attribution{--- Pink Floyd}
\poemsubtitle{z.}
\begin{stanza}
é um número z\verseline
simétrico\verseline
no alfabeto\verseline
da letra \textsc{z} (dois) \rule{4.25cm}{0.4pt}\verseline
\rule{4cm}{0.4pt} chega de conversa\verseline
vamos direto\verseline
ao que interessa\verseline
não sei o que é tambem\verseline
mas tudo bem\verseline
vamos\verseline
preciso de força\verseline
pelo menos\verseline
para escrever até o fim deste livro\verseline
\rule{2cm}{0.4pt} não para viver depois disso.
\end{stanza}
\end{poem}

\begin{center}
*\\
*\\
*\\
*\\
*\\
*\\
*
\end{center}
\clearpage

\vspace*{\fill}
$\star$\\
\vspace{3ex}
\begin{minipage}{\textwidth}
\center{{\huge{LIVRO} {\Huge \raisebox{1.5ex}{\rotatebox{180}{C }}}}}
\center{\textsc{
O que se diz\\
em verdade\\
a respeito da vida
}}\\
\center{\raisebox{1ex}{\rule{4cm}{0.4pt}}}
\center{\textsc{Gritos de dor}}
\center{\raisebox{1ex}{\rule{4cm}{0.4pt}}}
\center{\textsc{Lamentos}}
\center{\raisebox{1ex}{\rule{4cm}{0.4pt}}}
\center{\textsc{E outras mentiras}}
\center{(e citações apócrifas)}
\end{minipage}
\vspace{3ex}\\
$\star$
\vspace{\fill}

\clearpage

\vspace*{\fill}
OU
\vspace{\fill}

\clearpage

\newlength{\wa}
\settowidth{\wa}{\large{black states in the hour of}}

\vspace*{\fill}
\begin{large}\center{Escritos sobre \textsc{labirintos}}\end{large}
\vspace{4ex}
\center{i.e.}
\vspace{4ex}
\begin{large}
\center{
A exatidão\\
\begin{tabular}{ll}
\makebox[\wa][r]{do} & Caos\\
\text{} & Caos\\
\text{} & Caos\\
\text{} & Caos\\
\text{} & Caos\\
black states in the hour of & Chaos\\
\text{} & Kháos\\
\text{} & Caus\\
\end{tabular}
}
\end{large}
\vspace{\fill}

\clearpage

\justify{\huge{Parte-se talvez de uma reconciliação. Ou a conciliação pela primeira vez. Daqui-lo chamado fundo com aquilo chamado forma. Algo que talvez se possa chamar poema. Depois de muitos meses.\\
Trata-se de enigmas obscuros. Que é subtítulo de outro livro. Aqui há labirintos e a morte ciclica. Aqui anda-se em círculos. --- Se no labirinto de Abenjacan, o Bokari, chegava-se ao centro ao andar sempre para a direita, é porque aquele era um labirinto irregular. Não este. Este se pretende regular, e ir sempre para o mesmo lado é andar em círculos.\\
\begin{flushright}--- 27.8.2015\end{flushright}}}

\clearpage

\vspace*{\fill}
\begin{minipage}{\textwidth}
\large
\center{O}
\center{LABIRINTO\\
\qquad DENTRO\\
\qquad \quad DO}
\center{MONSTRO}
\end{minipage}
\vspace{\fill}


\clearpage

\vspace*{\fill}
\begin{minipage}{\textwidth}
\center
{
“Então eu havia me perdido num labirinto de perguntas [...]”
}\\
\vspace{2ex}
\begin{flushright}\textit{--- A Paixão Segundo G.H., Clarice Lispector}\end{flushright}
\end{minipage}
\vspace{\fill}
\clearpage
\begin{poem}
\begin{stanza}
os sinais nas paredes\verseline
do labirinto\verseline
são não nomes
\end{stanza}
\begin{stanza}
uma confusão de querer\verseline
dizeres a luz\verseline
do dia \qquad é falável\verseline
que a epifania\verseline
é a verdade em pedaços
\end{stanza}
\begin{stanza}
porque sentir o por dentro\verseline
caótico\verseline
é o achar a si\verseline
em cacos
\end{stanza}
\begin{stanza}
ter fragmentos de eu\verseline
perdidos
\end{stanza}
\begin{stanza}
e não saber mais\verseline
onde eles se encaixam ou\verseline
se já encaixaram
\end{stanza}
\begin{stanza}
e quando dizem a palavra\verseline
nome \qquad já não dizem\verseline
a pessoa
\end{stanza}
\begin{stanza}
porque agora\verseline
a pessoa se torna\verseline
impossível\verseline
e nada mais pode\verseline
que desexistir
\end{stanza}
\end{poem}
\clearpage
\begin{poem}
\sequencetitle{NA COZINHA}
\begin{stanza}
o pano de prato\verseline
em cima do balcão\verseline
(e um jeito certo\verseline
de estar desarrumado)\verseline
é o primeiro alvo\verseline
do olho que chega\verseline
a esse concerto do desconcerto\verseline
muito bem ensaiado\verseline
como se tirando uma peça\verseline
do seu lugar errado\verseline
não se arrumasse nada\verseline
mas destruísse toda\verseline
a premeditação\verseline
da desordem ordeira\verseline
bagunça familiar
\end{stanza}
\end{poem}
\clearpage
\begin{poem}
\sequencetitle{RASCUNHO}
\begin{stanza}
se libertar como\verseline
cometer um assassinato\verseline
\qquad \qquad do que pesa\verseline
\qquad \qquad e te encurva
\end{stanza}
\begin{stanza}
ou como\verseline
florescer no meio da agonia\verseline
de viver
\end{stanza}
\begin{stanza}
se seu nome\verseline
\qquad \qquad florescente\verseline
fosse agonia isso\verseline
seria mais que uma rima\verseline
seria a \verseline
própria vida
\end{stanza}
\end{poem}
\clearpage
\begin{poem}
\sequencetitle{WHEN I WAS MYSELF}
\begin{stanza}
a lua não merece ser olhada\verseline
nem nenhuma foto\verseline
mas não há opção
\end{stanza}
\begin{stanza}
os acontecimentos me fizeram\verseline
tão pobre e é tudo\verseline
vazio demais
\end{stanza}
\begin{stanza}
não é sofrer se é por nada
\end{stanza}
\begin{stanza}
caminhando nos destroços\verseline
da última grande coisa\verseline
não dou mais a mínima 
\end{stanza}
\attribution{--- 6.11.2015}
\end{poem}
\clearpage

\newpage
\thispagestyle{empty}
\mbox{}
\newpage


\vspace*{\fill}
\thispagestyle{empty}
\begin{minipage}{\textwidth}
\center
{
ONE OF THESE\\
DAYS\\
I’M GOING TO\\
CUT YOU\\
INTO LITTLE\\
PIECES \small{$\star \star \star$}
}
\end{minipage}
\vspace{\fill}

\clearpage

\thispagestyle{empty}
\center{Typeset in \LaTeX}
\center{Written in the \textsc{Computer Modern} font}
\center{Edited by Rocha}

\end{document}