\begin{poem}
\sequencetitle{3. Lá fora}
\begin{stanza}
andar na rua com livros\verseline
debaixo do braço\verseline
porque a leitura isso e aquilo\verseline
\qquad mas o mundo ainda escorre\verseline
\qquad por entre os dedos\verseline
\qquad \qquad pra que serve a mão\verseline
\qquad \qquad que não pode segurar\verseline
\qquad \qquad as coisas \quad suarazão?\verseline
os olhares e pernas cruzados\verseline
nos bancos em que\verseline
os cafés ao lado esfriam\verseline
\qquad esfriam as vontades\verseline
\qquad depois de alguns segundos\verseline
já é rotina andar no automático\verseline
aceitar a verdade absoluta\verseline
do outdoor da camiseta\verseline
\qquad conseguir um prazer\verseline
\qquad pouco ou nenhum\verseline
\qquad explorando a autotortura\verseline
o mendigo que cursou teologia\verseline
e pede um dinheiro pra cachaça\verseline
\qquad \qquad \qquad ganha\verseline
\qquad é um prêmio pela sinceridade\verseline
\qquad \qquad tem gente\verseline
\qquad \qquad que nunca diz o que quer\verseline
\qquad \qquad e não vale um cuspe\verseline
os carros correm\verseline
são o moto perpétuo\verseline
o motor que produz\verseline
o ar puro\verseline
com cheiro de cidade\verseline
\qquad a aparente desordem\verseline
\qquad dessa tarde\verseline
\qquad não é mais que um outro jeito\verseline
\qquad de arrumar as coisas\verseline
que eu morra\verseline
mas nunca veja demolida\verseline
aesquina onde eu sorri\verseline
e ganhei um sorriso\verseline
que esta queimando até agora
\end{stanza}
\end{poem}